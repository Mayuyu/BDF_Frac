% !TEX TS-program = pdflatex
% !TEX encoding = UTF-8 Unicode

% This is a simple template for a LaTeX document using the "article" class.
% See "book", "report", "letter" for other types of document.

\documentclass[11pt]{article} % use larger type; default would be 10pt

\usepackage[utf8]{inputenc} 
\usepackage{geometry} % to change the page dimensions
\geometry{a4paper} 

\usepackage{mathrsfs, graphics,amssymb,amsmath,subeqnarray,setspace,graphicx,amsthm,epstopdf,subfig,color}
\usepackage{booktabs,array,paralist,verbatim} 
\usepackage{enumitem}

%%% HEADERS & FOOTERS
\usepackage{fancyhdr} % This should be set AFTER setting up the page geometry
\pagestyle{fancy} % options: empty , plain , fancy
\renewcommand{\headrulewidth}{0pt} % customise the layout...
\lhead{}\chead{}\rhead{}
\lfoot{}\cfoot{\thepage}\rfoot{}

%%% SECTION TITLE APPEARANCE
\usepackage{sectsty}
\allsectionsfont{\sffamily\mdseries\upshape} % (See the fntguide.pdf for font help)
% (This matches ConTeXt defaults)

\usepackage[nottoc,notlof,notlot]{tocbibind} 
\usepackage[titles,subfigure]{tocloft} 

\setlist[enumerate]{leftmargin=*}

\renewcommand{\cftsecfont}{\rmfamily\mdseries\upshape}
\renewcommand{\cftsecpagefont}{\rmfamily\mdseries\upshape} % No bold!
\newcommand{\tcb}[1]{\textcolor{blue}{#1}}
\newcommand{\tcr}[1]{\textcolor{red}{#1}}

\newtheorem{thm}{Theorem}
\newtheorem{lmm}{Lemma}
\newtheorem{rmk}{Remark}
\newtheorem{pro}{Proposition}
\newtheorem{cor}{Corollary}
\newtheorem{defi}{Definition}
%%% END Article customizations

%%% The "real" document content comes below...

\title{Second order BDF method for FODE and FKS equations}
\author{The Authors}
%\date{} % Activate to display a given date or no date (if empty),
         % otherwise the current date is printed 

\begin{document}
\maketitle

\section{Introduction}

\section{The schemes and main results}

\tcb{Some preliminaries:}


Below $a\in \mathbb{R}^{\mathbb{N}}$ is a sequence of the form $a=(a_0, a_1, \ldots, a_m, \ldots)$ where $a_m\in\mathbb{R}$.
For $a\in\mathbb{R}^{\mathbb{N}}$, the generating function is given by $$
F_a(z)=\sum_{m=0}^{\infty}a_m z^m.
$$
For $a, c\in \mathbb{R}^{\mathbb{N}}$, we define the convolution $a*c$ as $$
(a*c)_n=\sum_{m=0}^{n}a_m c_{n-m},
$$
and clearly $F_{a*c}=F_a(z)F_c(z)$. The convolution identity is $\delta_d=(1,0,\ldots)$ whose generating function is $1$.
For a sequence $b$($b_0\neq 0$) with generating function $F(z)$, the generation function of the convolution inverse of $b^{(-1)}$ is given by $1/F(z)$.

\subsection{Numerical approximation of the Caputo derivative}
Review of the first order scheme
\\
The second order scheme:

Consider a time step $k$ and $t^n=nk$. We assume that $T=Nk$ is a fixed constant. 
Let $u=(u_0, u_1, u_2, \ldots)\in \mathbb{R}^{\mathbb{N}}$ be the sequence of nodal values of $u$.

Below, we given a second order backward differentiation scheme for $D_c^{\gamma}u$. For the discussions from here on, we will assume that $u_0$ and $u_1$ are given. 

For $n\ge 1$, the derivative is computed by $$
D_c^{\gamma}u(t^{n+1})\approx
k^{-\gamma}(b^{n+1}*u)_{n+1}=:\mathcal{D}_k u.
$$
Here $b^{n+1}=(b_m^{n+1})_{m=0}^{n+1}$ is a sequence with $n+2$ components. If we set $b_m^{n+1}=0, m\ge n+2$, then $b^{n+1}\in\mathbb{R}^{\mathbb{N}}$. $b^{n+1}*u$ can be defined. $(b^{n+1}*u)_{n+1}=\sum_{m=0}^{n+1}b_{m}^{n+1}u_{n+1-m}$ is the $n+1$-th component of this convolution.

The sequences $b^{n+1}$ are given as follows:

For $n=1$,
\begin{gather*}
\Gamma(2-\gamma)b_{0}^{2}=-\frac{1}{2}2^{1-\gamma}+\frac{1}{2-\gamma}2^{2-\gamma}\\
\Gamma(2-\gamma)b_{1}^{2}=2\cdot 2^{1-\gamma}-\frac{2}{2-\gamma}2^{2-\gamma}\\
\Gamma(2-\gamma)b_{2}^{2}=-\frac{3}{2}2^{1-\gamma}+\frac{1}{2-\gamma}2^{2-\gamma}
\end{gather*}

For $n=2$,
\begin{gather*}
\Gamma(2-\gamma)b_{0}^{3}=\frac{1}{2}+\frac{1}{2-\gamma}\\
\Gamma(2-\gamma)b_{1}^{3}
=-\frac{3}{2}-\frac{3}{2-\gamma}-\frac{1}{2}3^{1-\gamma}
+\frac{1}{2-\gamma}3^{2-\gamma}\\
\Gamma(2-\gamma)b_{2}^{3}=\frac{3}{2}+2\cdot 3^{1-\gamma}
+\frac{3}{2-\gamma}-\frac{2}{2-\gamma}3^{2-\gamma}\\
\Gamma(2-\gamma)b_{3}^{3}=-\frac{1}{2}
-\frac{3}{2}3^{1-\gamma}+\frac{1}{2-\gamma}(3^{2-\gamma}-1)
\end{gather*}

For $n\ge 3$:

\begin{gather}
\begin{array}{c}
\Gamma(2-\gamma)b_{0}^{n+1}=\frac{1}{2}+\frac{1}{2-\gamma},\\
\Gamma(2-\gamma)b_1^{n+1}=-\frac{3}{2}+2^{1-\gamma}\frac{1}{2}
-\frac{3}{2-\gamma}+\frac{1}{2-\gamma}2^{2-\gamma}
 \end{array}
\end{gather}

When $2\le m\le n-2$, we have $\Gamma(2-\gamma)b_{m}^{n+1}=h(m)$ where the function $h$ is given by
\begin{multline}
h(m)=-\frac{1}{2}(m-2)^{1-\gamma}-\frac{3}{2}m^{1-\gamma}+\frac{3}{2}(m-1)^{1-\gamma}+\frac{1}{2}(m+1)^{1-\gamma}\\
+\frac{1}{2-\gamma}(m+1)^{2-\gamma}
-\frac{3}{2-\gamma}m^{2-\gamma}
+\frac{3}{2-\gamma}(m-1)^{2-\gamma}
-\frac{1}{2-\gamma}(m-2)^{2-\gamma}
\end{multline}
For the last three coefficients,
\begin{gather}
\Gamma(2-\gamma)b_{n-1}^{n+1}
=-\frac{1}{2}n^{1-\gamma}-\frac{1}{2}(n+1)^{1-\gamma}
-\frac{1}{2-\gamma}[n^{2-\gamma}-(n+1)^{2-\gamma}]
+h(n-1)\\
\Gamma(2-\gamma)b_n^{n+1}=h(n)+\frac{3}{2}n^{1-\gamma}+\frac{3}{2}(n+1)^{1-\gamma}
+\frac{3}{2-\gamma}[n^{2-\gamma}-(n+1)^{2-\gamma}]\\
\Gamma(2-\gamma)b_{n+1}^{n+1}
=-\frac{1}{2}(n-1)^{1-\gamma}-\frac{3}{2}(n+1)^{1-\gamma}
-\frac{1}{2-\gamma}(n-1)^{2-\gamma}+\frac{1}{2-\gamma}(n+1)^{2-\gamma}
\end{gather}


\begin{rmk}
In real simulation, we can compute $u^1$, for example using a first order scheme. The error for $u^1$ will then be $O(k^2)$ \tcr{Need to confirm.}. This is not as good as the $O(k^{3-\gamma})$ estimation below. An alternative way is to $L_1(t)$ on both $I_0, I_1$. The relations 
\begin{gather*}
D_c^{\gamma}u(t^1)\approx k^{-\gamma}\frac{1}{\Gamma(2-\gamma)}[(-\frac{1}{2}
+\frac{1}{2-\gamma})u_2+(2-\frac{2}{2-\gamma})u_1+(-\frac{3}{2}+\frac{1}{2-\gamma})u_0],\\
D_c^{\gamma}u(t^2)\approx k^{-\gamma}b^{2}*u
\end{gather*}
then form a system of equations for $u^1, u^2$. We then compute them all at once. The local truncation error is $O(k^{3-\gamma})$.
\end{rmk}

\subsection{Main results on the second order BDF method for FODE}
stability and accuracy results of the BDF method for $D^\gamma_c u = \lambda u$. \\
stability results on variant BDF method.

\begin{thm}
Suppose $u\in C^3([0,\infty), \mathbb{R})$. Then, $\exists C>0$ independent of $\gamma$ and $n$ such that $\forall 1\le  n\le N-1$, $$
|D_c^{\gamma}u(t^{n+1})-k^{-\gamma}(b^{n+1}*u)_{n+1}|\le Ck^{3-\gamma}.
$$
\end{thm}

If we apply the scheme on the model equation $D_c^{\gamma}u=\lambda u$ and define $z=k^{\gamma}\lambda$
we have  $$
(b_0^n-z)+\sum_{m=1}^n b_m^n \zeta^n_m=0.
$$
where $\zeta^m_n=u^{n-m}/u^n$. The characteristic polynomial is $(b_0^n-z)+\sum_{m=1}^n b_m^n \eta^m=0.$
As we shall see in Theorem .. below, there exists $b\in\mathbb{R}^{\mathbb{N}}$, such that $b^{n+1}_m=b_m$ for $m\le n-2$, and $b^{n+1}\to b$ in $l^p, p\ge 1$. Hence, when $n$ is sufficiently large, this relation is roughly 
$$
f(z, \eta)=(b_0-z)+\sum_{m=1}^{\infty} b_m \eta^m=0.
$$
$f=F_b(\eta)-z$. Clearly, we want $\eta\ge 1$. In other words, we need $f(z, \eta)$ to have no zeros in the unit disk. 

\begin{defi}
The stability region is $D=\{z\in\mathbb{C}: f(z, \eta)\text{ satisfies the root property}\}$
\end{defi}

\begin{defi}
The scheme is called $A$-stable if the stability region contains the whole left half plane. 
\end{defi}
Clearly, the $A$-stability is reduced to study of the generating function of $b$.
\begin{thm}
The second backward differentiation $\mathcal{D}_k$ is $A$-stable. 
\end{thm}

\tcr{Add the result saying that if $\langle u^{n+1}, \cdot\rangle\le 0$, then it is absolutely stable. The norm will not grow, as in the new reference.}

\begin{thm}
Consider the FODE $D_c^{\gamma}u=A(u)$. Suppose $A_ku$ is a second order approximation for $A(u)$ such that $\exists\lambda\ge 0$, $$
\langle u^{n+1}, (A_ku)^{n+1}\rangle \le \lambda(\|u^{n}\|^2+\|u^{n-1}\|^2).
$$
The scheme given by
$$
(\mathcal{D}_ku)^{n+1}=(A_ku)^{n+1}, \ n\ge 1
$$
is $l^2$-stable, or in other words, $$
\sum_{m=0}^n |u_m|^2k\le C(T)E_{\gamma}(C_1(nk)^{\gamma})
$$
for all $n\le N$, where $E_{\gamma}$ is the Mittag-Leffler function.\tcr{Need to make precise the constants}
\end{thm}



\subsection{Main results on the first/second order method for FKS}
stability.\\
small data condition?

Consider the following fractional Keller-Segel equation:
\begin{gather*}
D_c^{\gamma}\rho=\Delta\rho-\nabla\cdot(\rho\nabla c)\\
-\Delta c=\rho
\end{gather*}

\begin{pro}
The first order scheme for the KS is stable with small initial data.
\end{pro}


\begin{pro}
The second order scheme for the KS is $l^2(L^2(\mathbb{R}^2))$-stable if the initial data is small.
\end{pro}


By the sign properties of of $b^{n+1}$, we have 
\begin{pro}
For the FKS and the second order scheme with five point.. \tcr{to complete},
$\exists \bar{\gamma}_0\in (0, 1)$, if $\gamma\le\bar{\gamma}_0$ and $u_0\ge 0, u_1\ge 0$, then $u_m\ge 0$ for all $m\ge 0$.
\end{pro}

\section{Construction of the numerical approximation of the Caputo derivative}
derivation.  \\
estimates on the coefficients.

To be convenient, we denote $I_m=[t^m, t^{m+1}]$.

Suppose $u_m$ is the nodal value of $u$ at $t^m$. The Lagrange interpolation on $I_m$ is given by:
$$
L_m(t)=u_{m-1}\frac{(t-t^m)(t-t^{m+1})}{2k^2}-u_m\frac{(t-t^{m-1})(t-t^{m+1})}{k^2}+u_{m+1}\frac{(t-t^m)(t-t^{m-1})}{2k^2}
$$
Hence, we use the approximation for $m\ge 1$
\begin{gather}
u'(t)\approx L_m'(t)=u_{m-1}\frac{t-t^{m+1/2}}{k^2}-2u_m\frac{t-t^m}{k^2}+u_{m+1}\frac{t-t^{m-1/2}}{k^2},\ t\in I_m.
\end{gather}
For $t\in I_0$, $L_0$ will be the quadratic approximation using $u_0, u_1, u_2$, i.e., we set $L_0=L_1$.

The numerical scheme is given by
\begin{gather}
\mathcal{D}_ku(t^{n+1})\approx\frac{1}{\Gamma(1-\gamma)}\sum_{m=0}^n
\int_{I_m}\frac{L_m'(s)}{(t^{n+1}-s)^{\gamma}}ds=k^{-\gamma}\sum_{m=0}^{n+1}b^{n+1}_m u_{n+1-m}.
\end{gather}
We call this formula `backward differentiation' since $b_0^{n+1}>0$. As we shall see the scheme tends to BDF2 as $\gamma\to 1$.

We define the coefficients $C_m^p$ for $m\ge 1$ through: $$
k^{\gamma}(1-\gamma)\int_{I_m}\frac{L_m'(s)}{(t^{n+1}-s)^{\gamma}}ds=C_m^{m+1}u_{m+1}+C_m^mu_m+C_{m}^{m-1}u_{m-1}.
$$
Hence,
\begin{gather*}
\begin{array}{cl}
C_m^{m+1}=&-\frac{3}{2}(n-m)^{1-\gamma}+\frac{1}{2}(n-m+1)^{1-\gamma}
-\frac{1}{2-\gamma}(n-m)^{2-\gamma}+\frac{1}{2-\gamma}(n-m+1)^{2-\gamma},\\
C_m^m=&2(n-m)^{1-\gamma}+\frac{2}{2-\gamma}(n-m)^{2-\gamma}
-\frac{2}{2-\gamma}(n-m+1)^{2-\gamma},\\
C_m^{m-1}=&-\frac{1}{2}(n-m)^{1-\gamma}-\frac{1}{2}(n-m+1)^{1-\gamma}
-\frac{1}{2-\gamma}(n-m)^{2-\gamma}+\frac{1}{2-\gamma}(n-m+1)^{2-\gamma}.
\end{array}
\end{gather*}
These are only valid for $m\ge 1$.

For $m=0$, we compute $k^{\gamma}(1-\gamma)\int_0^{t^1}\frac{L_1'(t)}{(t^{n+1}-s)^{\gamma}}ds$ where
\begin{gather*}
L_1'(t)=u_{0}\frac{t-t^{3/2}}{k^2}-2u_1\frac{t-t^1}{k^2}+u_{2}\frac{t-t^{1/2}}{k^2}.
\end{gather*}
and find
\begin{gather*}
C_0^2=-\frac{1}{2}n^{1-\gamma}-\frac{1}{2}(n+1)^{1-\gamma}
-\frac{1}{2-\gamma}[n^{2-\gamma}-(n+1)^{2-\gamma}],, \\
C_0^1=2(n+1)^{1-\gamma}
+\frac{2}{2-\gamma}[n^{2-\gamma}-(n+1)^{2-\gamma}], \\
C_0^0=\frac{1}{2}n^{1-\gamma}-\frac{3}{2}(n+1)^{1-\gamma}
-\frac{1}{2-\gamma}[n^{2-\gamma}-(n+1)^{2-\gamma}]
\end{gather*}
Hence, by the definition, we have
\begin{multline*}
D_c^{\gamma}u(t^{n+1})\approx \frac{1}{k^{\gamma}}\sum_{m=0}^{n+1}b_{m}^{n+1}u_{n+1-m}=\\
\frac{1}{k^{\gamma}\Gamma(2-\gamma)}(C_0^0u_0+C_0^1u_1+C_0^2u_2+\sum_{m=1}^n(C_m^{m-1}u_{m-1}
+C_m^{m}u_m+C_m^{m+1}u_{m+1}))
\end{multline*}
Using the explicit formulas, we obtain the coefficients listed in Section...


\begin{thm}
We have the following claims:
\begin{enumerate}
\item 
For all $n\ge 1$, $$
\sum_{m=0}^{n+1}b_m^{n+1}=0.
$$ 
For all $n\ge 1$, $b_0^{n+1}>0$, $b_1^{n+1}<0$. For $n\ge 2$, $b_m^{n+1}<0$, $m\ge 3$. $\exists \bar{\gamma}_0\in (0, 1)$ and $\bar{\gamma}_1\in(\bar{\gamma}_0, 1)$ such that $b_2^{n+1}\le 0$ for $n\ge 1, \gamma<\bar{\gamma}_0$ and $b_2^{n+1}>0$ for $n\ge 1, \gamma>\bar{\gamma}_1$.


There exists a sequence $b\in \mathbb{R}^{\mathbb{N}}$ such that if $n\ge 2$, $b_m^{n+1}=b_m$ for $m\le n-2$.

\item 
When $n\ge 1$, as $\gamma\to 1$, $$
b_0^{n+1}\to 3/2, b_1^{n+1}\to -2, b_2^{n+1}\to 1/2, b_m^{n+1}\to 0
$$
and the scheme tends to BDF2; as $\gamma\to 0$, $$
b_0^{n+1}\to 1, b_m^{n+1}\to 0 ,b_{n+1}^{n+1}\to -1.
$$
\item As $n\to \infty$, $b_{n-1}^{n+1}=O(n^{-1-\gamma})$ ,$b_n^{n+1}=O(n^{-1-\gamma})$ and $b_{n+1}^{n+1}=O(n^{-\gamma})$. $b^{n+1}\to b$ pointwise and in $l^p$, $p\ge 1$. 

For the limiting sequence $b$, $b_0>0$, $b_m<0$
for $m\ge 1, m\neq 2$, and $$
\sum_m b_m=0,\ 
b_0\ge -\frac{3}{4}b_1,\ \ -b_1\ge 4b_2.
$$
$\exists \theta>0$, $(4b_0+3b_1)/(1-\gamma)\le \theta$ for all $\gamma\in (0, 1)$. $\exists\gamma_0\in (0,1)$ such that $b_2\le 0$ if $\gamma\le \gamma_0$ and $b_2>0$ for $\gamma>\gamma_0$. 
When $m\ge 3$, $b_m$ increases to zero and $b_m=\frac{1}{\Gamma(-\gamma)}\frac{1}{m^{1+\gamma}}\left(1+O(\frac{1}{m})\right)$ as $m\to\infty$.
\end{enumerate}
\end{thm}


To make the statement clean. The proof of this theorem is put in the appendix.




\section{Stability and convergence analysis}


\subsection{Local truncation error}
Suppose $u$ is $C^3$. We know estimate the local truncation error.

Proof of Theorem:

By the Taylor formula $$
f(t)=f(s)+(t-s)f'(s)+\frac{1}{2}(t-s)^2f''(s)+\int_s^t\frac{1}{2}(t-\tau)^2 f'''(\tau)d\tau,
$$
we find 
\begin{multline}
R_m(s)=L_m'(s)-u_s(s)=\frac{s-t^{m+1/2}}{k^2}\int_s^{t^{m-1}}\frac{1}{2}(t^{m-1}-\tau)^2 u'''(\tau)d\tau\\
-2\frac{s-t^m}{k^2}\int_s^{t^m}\frac{1}{2}(t^m-\tau)^2 u'''(\tau)d\tau
+\frac{s-t^{m-1/2}}{k^2}\int_s^{t^{m+1}}\frac{1}{2}(t^{m+1}-\tau)^2 u'''(\tau)d\tau
\end{multline}
The total error will be $$
r=\frac{1}{\Gamma(1-\gamma)}\left(r_0+\sum_{m=1}^n\int_{I_m}\frac{R_m(s)}{(t^{n+1}-s)^{\gamma}}ds\right)
$$
where $$
r_0=\int_{I_0}\frac{R_1(s)}{(t^{n+1}-s)^{\gamma}}ds.
$$

Direct estimate shows $$
r_0\le Ck^2\int_0^k\frac{1}{((n+1)k-s)^{\gamma}}ds\le C_1k^{3-\gamma}.
$$
where $C_1$ is independent of $n,\gamma$. If we use the same estimate, $|R_m|\le Ck^2$, then we obtain $$
r\le \frac{1}{\Gamma(1-\gamma)}(r_0+Ck^2\int_{t^1}^{t^{n+1}}\frac{1}{(t^{n+1}-s)^{\gamma}}ds)\le \frac{C}{\Gamma(2-\gamma)}T^{1-\gamma}k^2
$$
The scheme is at least uniformly second order.  

Better estimates 
\begin{gather*}
\frac{1}{\Gamma(1-\gamma)}\int_{t^m}^{t^{m+1}}\frac{R_m(s)}{(t^{n+1}-s)^{\gamma}}ds=I_1^m+I_2^m
\end{gather*}
where $$
I_1^m=\frac{1}{\Gamma(1-\gamma)}\frac{1}{2k^2}\int_{t^{m-1}}^{t^{m}}
u'''(\tau)(t^{m+1}-\tau)^2\int_{t^m}^{t^{m+1}}\frac{t^{m+1/2}-s}{(t^{n+1}-s)^{\gamma}}dsd\tau,
$$
and $I_2^m=\frac{1}{2k^2\Gamma(1-\gamma)}\int_{t^m}^{t^{m+1}}u'''(\tau)P_m(\tau)d\tau$,
\begin{multline*}
P_m(\tau)=(t^{m-1}-\tau)^2\int_{\tau}^{t^{m+1}}\frac{t^{m+1/2}-s}{(t^{n+1}-s)^{\gamma}}ds\\
-2(t^m-\tau)^2\int_{\tau}^{t^{m+1}}\frac{t^{m}-s}{(t^{n+1}-s)^{\gamma}}ds
+(t^{m+1}-\tau)^2\int_{t^m}^{\tau}\frac{s-t^{m-1/2}}{(t^{n+1}-s)^{\gamma}}ds.
\end{multline*}

Note that $\int_{t^m}^{t^{m+1}}\frac{t^{m+1/2}-s}{(t^{n+1}-s)^{\gamma}}ds$
is negative for all $m$, we find $$
|\sum_m I_1^m|\le \sum_m \frac{Ck}{\Gamma(1-\gamma)}\int_{t^m}^{t^{m+1}}\frac{s-t^{m+1/2}}{(t^{n+1}-s)^{\gamma}}ds
$$
where $C\sim \sup|u'''|$ is independent of $n,m,\gamma$.
\begin{multline*}
\int_{t^m}^{t^{m+1}}\frac{s-t^{m+1/2}}{(t^{n+1}-s)^{\gamma}}ds
=\frac{k^{2-\gamma}}{1-\gamma}[
\frac{1}{2-\gamma}((n-m+1)^{2-\gamma}-(n-m)^{2-\gamma})
-\frac{1}{2}((n-m+1)^{1-\gamma}+(n-m)^{1-\gamma})]
\end{multline*}
This number $\le$ $\frac{Ck^{2-\gamma}}{1-\gamma}(n-m+1/2)^{-1-\gamma}$ by Taylor expansion about $n-m+1/2$, where $C=\sup_{\gamma}\gamma(1-\gamma)$ is independent of $m,\gamma, n$. Hence, $$
|\sum_{m\ge 1} I_1^m|\le \sum_{m=1}^n
\frac{1}{\Gamma(1-\gamma)}\frac{Ck^{3-\gamma}}{1-\gamma}(n-m+1/2)^{-1-\gamma}\le C_1k^{3-\gamma}
$$
where $C_1$ is independent of $\gamma$ and $n$.

For $I_2^m$, in the appendix, we show that $P_m(\tau)\le 0$ for all $m\ge 1$ and $\tau\in I_m$. Applying the integral mean value theorem, we obtain  $\int_{t^m}^{t^{m+1}}u'''(\tau)P_m(\tau)d\tau
=u'''(\xi) \int_{I_m}P_m(\tau)d\tau$. Direct computation shows
\begin{multline*}
(1-\gamma)\int P_m(\tau)d\tau=\frac{k^3}{2}k^{2-\gamma}(n-m)^{1-\gamma}+
\frac{5}{3}\frac{1}{2-\gamma}(n-m)^{2-\gamma}\\
+(\frac{k^3}{3})k^{2-\gamma}[\frac{1}{2}(n-m+1)^{1-\gamma}+\frac{1}{2-\gamma}(n-m+1)^{2-\gamma}]\\
+\frac{2k^2}{(2-\gamma)(3-\gamma)}(n-m)^{3-\gamma}k^{3-\gamma}
-\frac{2k^2}{(2-\gamma)(3-\gamma)}(n-m+1)^{3-\gamma}k^{3-\gamma}
\end{multline*}
We do Taylor expansion about $n-m+1/2$. We find $2-\gamma$, $1-\gamma$, $-\gamma$ all cancel out. 
The nonzero power will be $(n-m+1/2)^{-1-\gamma}$, and hence summable.

Hence,  $$
|\sum_{m\ge 1}I_2^m|\le\sum_{1\le m\le n}\frac{1}{k^2\Gamma(1-\gamma)} \frac{C_2}{1-\gamma}k^{5-\gamma}(n-m+1/2)^{-1-\gamma}
$$
Hence, $$
r\le \frac{1}{\Gamma(1-\gamma)}r_0+C_3k^{3-\gamma}\le Ck^{3-\gamma},
$$
where $C$ is uniform for $\gamma\in [0,1]$ since $\Gamma(1-\gamma)\ge 1$.


\subsection{A-stability}
 In the case $b_2\le 0$, the scheme is $A$-stable. This is because all coefficients except $b_0$ are all negative. $|b_0-z|\le \sum_{m=1}^{\infty}(-b_m^{\infty})|\eta|^m\le b_0$
if $|\eta|\le 1$. 

Now, assume $b_2>0$. It would suffice to show $\inf_{z\in D(0,1)}Re(f(\eta, 0))\ge 0$. Recall that $b_1<0, b_2>0$

Since $-|\sum_{m=3}^{\infty}b_m\eta^m|\ge -\sum_{m=3}^{\infty}|b_m|
=\sum_{m=3}^{\infty}b_m=-(b_0+b_1+b_2)$, we find 
 $Re(f(\eta,0))\ge b_0+Re(b_1\eta+b_2\eta^2)-(b_0+b_1+b_2)$

Setting $\eta=r\cos\theta$ results in the function $$
h(r,\theta)=|b_1|(1-r\cos\theta)-b_2(1-r^2\cos(2\theta)).
$$
This function is positive for $\cos\theta\le 0$ because $|b_1|\ge 2b_2$. For $\cos\theta\ge 0$, $h_r=-|b_1|\cos\theta+2b_2r\cos(2\theta)\le 0$ since $r\cos(2\theta)\le\cos(\theta)$. The minimum value is achieved on $r=1$. 

Set $h(\theta)=|b_1|(1-\cos\theta)-b_2(1-\cos(2\theta))$ with $0\le\theta\le \pi/2$. We find that $h(0)=0$, which is desired since $f(1,0)=0$. $h'=\sin\theta (|b_1|-4b_2\cos\theta)$. Since $|b_1|>4b_2$, $h>0$ when $\theta\neq 0$.  Hence, $h(\theta)>0$ for $0<\theta\le \pi/2$.

\subsection{Analysis of the FODE}

\begin{lmm}\label{lmm:convinv}
Suppose $b=\{b_n\}$ is a sequence satisfying $\sum_{n=0}^{\infty}b_n=0$
and $b_0>0$. Suppose further that $b_n=\frac{1}{\Gamma(-\gamma)}n^{-1-\gamma}(1+O(\frac{1}{n}))$, where $0<\gamma<1$. Let $F(z)$ be its generating function $F(z)=\sum_{n=0}^{\infty}b_n z^n$. Then, 
\begin{enumerate}
\item $$
F(z)-(1-z)^{\gamma}=(1-z)G(z)
$$
where $G(z)$ is bounded in the unit disk.
\item Let $b^{(-1)}$ be the convolution inverse of $b$. Its generatating function $F(z)^{-1}\sim (1-z)^{-\gamma}$ and thus $b^{(-1)}_n=[z^n]F(z)^{-1}\sim \frac{1}{\Gamma(\gamma)}n^{\gamma-1}$ 
\end{enumerate}
\end{lmm}
\begin{proof}
Let $H(z)=F(z)-(1-z)^{\gamma}$. Then, $H(1)=0$. Consider the Taylor series of $H$: $$
H(z)=\sum_{n=0}^{\infty}d_n z^n.
$$
By the asymptotic behavior of $b$ and the asymptotic behavior of the coefficients of $(1-z)^{\gamma}$, we find
$$d_n=O(\frac{1}{n^{2+\gamma}}).$$ 

Now consider $$
\frac{H(z)}{1-z}=\frac{H(z)-H(1)}{1-z}=\sum \frac{d_n(z^n-1)}{1-z}.
$$
Then, for each term, 
$|d_nz^n-d_n|(1-z)^{-1}\le Cnd_n$ where $C$ is independent of $n$ and $z\in D(0,1)$. Since $\sum n|d_n|$ converges. Hence, $G(z)=H(z)/(1-z)$ is bounded. The first claim is proved.


Using the first claim, we find that $$
F(z)^{-1}=(1-z)^{-\gamma}\frac{1}{1+(1-z)^{1-\gamma}G(z)}
$$
This implies that as $z\sim 1$, $F(z)^{-1}\sim (1-z)^{-\gamma}$.

By Corollary VI.1 in \cite{fs09}, we have 
$b^{(-1)}_n=[z^n]F(z)^{-1}\sim \frac{1}{\Gamma(\gamma)}n^{\gamma-1}$.
\end{proof}
By the second statement, we are able to conclude:
\begin{cor}
 If the sequence $b$ satisfies the conditions in Lemma \ref{lmm:convinv}, then $\exists C>0$ such that $|b_n^{(-1)}|\le C\frac{1}{\Gamma(\gamma)}(n+1)^{\gamma-1}$.  
\end{cor}
We will then use the notation $C_b$ to represent the sharp constant for this the sequence $b$.

The following discrete Gronwall inequality is important (\cite{dixon86}): 
\begin{lmm}
Let $a\in \mathbb{R}^{\mathbb{N}}$ and $a_i\ge 0, 0\le i\le N$. If $a$ satisfies $$
a_n\le B+\frac{M}{\Gamma(\gamma)}k^{\gamma}\sum_{m=0}^{n-1}(n-m)^{\gamma-1}a_m, \forall 0\le n\le N,
$$
where $B>0$ and $M>0$ are independent of $n, k, \gamma$, then, $$
a_n\le BE_{\gamma}(M(nk)^{\gamma}), 0\le n\le N.
$$
The function $E_{\gamma}$ is the Mittag-Leffler function.
\end{lmm}

Now, we prove an important lemma
\begin{lmm}\label{lmm:stabilitylmm}
Suppose $b=\{b_n\}$ is a sequence satisfying the properties above. Let $E=\{E^n\}$ be a nonnegative sequence. Let $b^{(-1)}$ be the convolution inverse of $b$, which has generating function $1/F(z)$. Let $k=T/N$ where $N$ is a big integer.

If one of the following two conditions is satisfied:

 (i). $\exists \gamma_n\ge 0,\beta\ge 0$, $\gamma_n\le C\frac{1}{\Gamma(1-\gamma)}k^{-\gamma}(1+n)^{-\gamma}$
 $$
k^{-\gamma}|(b*E)_{n+1}|\le \gamma_n+\beta E^n,\forall 0\le n\le N.
$$

(ii). $b^{(-1)}_n\ge 0$ and $\exists \gamma_n\ge0, \beta\ge0$, $\gamma_n\le C\frac{1}{\Gamma(1-\gamma)}k^{-\gamma}(1+n)^{-\gamma}$
$$
k^{-\gamma}(b*E)_{n+1}\le \gamma_n+\beta E^n,\forall 0\le n\le N.
$$
Then, $$
E^n\le (C_bb_0E_0/\Gamma(\gamma)+C_bC)E_{\gamma}(C_b\beta(nk)^{\gamma}),\ \forall 0\le n\le N.
$$
\end{lmm}
\begin{proof}
Let $H_n=k^{-\gamma}(b*E)_{n}$ and consider the sequence $H=\{H_n\}$. Then clearly, we have $E^n=k^{\gamma}(b^{(-1)}*H)_n$ for all $n\ge 0$.

Recall that $|b_n^{(-1)}|\le \frac{C_b}{\Gamma(\gamma)}(n+1)^{\gamma-1}$.  For $0\le n\le N$, if the first case happens, we have 
\begin{gather*}
E^n\le k^{\gamma}\sum_{i=0}^n |b_{n-j}^{(-1)}| |H_{j}|
\le |b_n^{(-1)}|b_0E_0+C_bk^{\gamma}\sum_{j=1}^{n} \frac{1}{\Gamma(\gamma)}(n-j+1)^{\gamma-1}
(\gamma_{j-1}+\beta E^{j-1})\\
\le (|b_n^{(-1)}|b_0E_0+C_bC) +k^{\gamma}\frac{C_b\beta}{\Gamma(\gamma)}\sum_{j=0}^{n-1}(n-j)^{\gamma-1}E^j,\ \forall 0\le n\le N
\end{gather*}
if the second case happens, then, we have 
\begin{gather*}
E^n\le k^{\gamma}\sum_{j=0}^n b_{n-j}^{(-1)} H_{j}
\le b_n^{(-1)}b_0E_0+C_bk^{\gamma}\sum_{j=1}^{n} \frac{1}{\Gamma(\gamma)}(n-j+1)^{\gamma-1}
(\gamma_{j-1}+\beta E^{j-1})\\
\le (b_n^{(-1)}b_0E_0+C_bC)+\frac{C_b\beta}{\Gamma(\gamma)}\sum_{j=0}^{n-1}(n-j)^{\gamma-1}E^j,\ \forall 0\le n\le N
\end{gather*}
Here, we have used the fact
\begin{multline*}
\frac{1}{\Gamma(\gamma)\Gamma(1-\gamma)}k^{\gamma}\sum_{j=0}^{n-1} (n-j)^{\gamma-1}k^{-\gamma}(1+j)^{-\gamma}
=\frac{1}{\Gamma(\gamma)\Gamma(1-\gamma)}\sum_{j=0}^{n-1} (1-\frac{j}{n})^{\gamma-1}(\frac{j+1}{n})^{-\gamma}\frac{1}{n}
\\
\le \frac{1}{\Gamma(\gamma)\Gamma(1-\gamma)}\sum_{j=0}^{n-1}\int_{j/n}^{(j+1)/n}(1-x)^{\gamma-1}x^{-\gamma}dx=1
\end{multline*}



The discrete Gronwall inequality guarantees that $E^n$ has the bound as indicated.
\end{proof}

We now prove theorem...
\begin{proof}

Now considering left hand side. The issue is that  $b_2$ may be positive. 

Consider that $n\ge 1$. Then, $b_m^{n+1}=b_m$ for $m\le n-2$. 

The key point is to write 
\begin{gather*}
b_0u^{n+1}+b_1u^n+b_2u^{n-1}
=-\frac{b_1}{2}(\frac{3}{2}u^{n+1}-2u^n+\frac{1}{2}u^{n-1})
+(b_0+\frac{3}{4}b_1)u^{n+1}+(b_2+\frac{b_1}{4})u^{n-1}
\end{gather*}

Define the new sequence $c_0^{n+1}=b_0+\frac{3}{4}b_1>0$, $c_1^{n+1}=0$,
$c_2^{n+1}=b_2+\frac{1}{4}b_2<0$, $c_m^{n+1}=b_m<0, m\ge 3$. $\sum_m c_m^{n+1}=0$ still holds. We also use $c$ to mean the limiting sequence for $c^{n+1}$. By the technique used in...
\begin{gather*}
\langle u^{n+1}, \sum_m c_m^{n+1}u^{n+1-m}\rangle\ge 
\frac{1}{2}\sum_m c_m^{n+1}\|u^{n+1-m}\|^2.
\end{gather*}
Further, 
\begin{multline*}
\langle u^{n+1}, \frac{3}{2}u^{n+1}-2u^n+\frac{1}{2}u^{n-1}\rangle
\ge \frac{1}{4}(\|u^{n+1}\|_2^2-\|u^n\|_2^2+\|2u^{n+1}-u^n\|_2^2-\|2u^n-u^{n-1}\|_2^2)
\end{multline*}
Since $\langle u^{n+1}, (\mathcal{D}_k u)^{n+1}\rangle\le \lambda(\|u^n\|^2+\|u^{n+1}\|^2)$ and $-b_1>0$, we have
\begin{multline*}
-\frac{b_1}{8}k^{-\gamma}(\|u^{n+1}\|_2^2-\|u^n\|_2^2+\|2u^{n+1}-u^n\|_2^2-\|2u^n-u^{n-1}\|_2^2)\\
+\frac{1}{2}k^{-\gamma}\left(c_0\|u^{n+1}\|^2
+c_2\|u^{n-1}\|^2
+\sum_{m=3}^{n+1}c_m\|u^{n+1-m}\|^2\right)\\
\le k^{-\gamma}(b_{n-1}-b_{n-1}^{n+1})\|u^2\|^2
 +k^{-\gamma}(b_{n}-b^{n+1}_n)\|u^1\|^2
+k^{-\gamma}(b_{n+1}-b_{n+1}^{n+1})\|u^0\|^2+\lambda(\|u^n\|^2+\|u^{n-1}\|^2)
\end{multline*}
This is valid for all $n\ge 1$.

Set $F_n=\frac{1}{n+1}\sum_{m=0}^n\|u^m\|_2^2$. 
\begin{multline*}
-\frac{b_1}{8(n+2)}k^{-\gamma}((\|u^{n+1}\|_2^2-\|u^1\|_2^2+\|2u^{n+1}-u^n\|_2^2-\|2u^1-u^{0}\|_2^2)\\
+\frac{1}{2}k^{-\gamma}\left(c_0F_{n+1}
+c_2\frac{n}{n+2}F_{n-1}
+\sum_{m=3}^{n+1}c_m\frac{n+2-m}{n+2}F_{n+1-m}\right)\\
\le k^{-\gamma}\frac{1}{n+2}\sum_{m=1}^{n}(b_{m-1}-b^{m+1}_{m-1})\|u^2\|^2
+k^{-\gamma}\frac{1}{n+2}(\sum_{m=1}^{n}(b_{m}-b^{m+1}_m)+\frac{c_0}{2})\|u^1\|^2\\
+k^{-\gamma}\frac{1}{n+2}(\frac{1}{2}c_0+\sum_{m=1}^{n}(b_{m+1}-b_{m+1}^{m+1}))\|u^0\|^2
+2\lambda F_n
\end{multline*}

By the explicit expressions
\begin{gather*}
b_{n-1}-b_{n-1}^{n+1}=\frac{1}{\Gamma(2-\gamma)}(\frac{1}{2}n^{1-\gamma}+\frac{1}{2}(n+1)^{1-\gamma}
+\frac{1}{2-\gamma}[n^{2-\gamma}-(n+1)^{2-\gamma}])\\
b_n-b_n^{n+1}=\frac{1}{\Gamma(2-\gamma)}(-\frac{3}{2}n^{1-\gamma}
-\frac{3}{2}(n+1)^{1-\gamma}-\frac{3}{2-\gamma}[n^{2-\gamma}-(n+1)^{2-\gamma}])\\
b_{n+1}-b_{n+1}^{n+1}=\frac{1}{\Gamma(2-\gamma)}(\frac{3}{2}n^{1-\gamma}+\frac{1}{2}(n+2)^{1-\gamma}
+\frac{1}{2-\gamma}[(n+2)^{2-\gamma}-4(n+1)^{2-\gamma}+3n^{2-\gamma}])
\end{gather*}
Taylor expanding $n^{1-\gamma}, (n+1)^{1-\gamma}$ at $x$ yields $$
\left|\int_n^{n+1}(\frac{1}{2}n^{1-\gamma}+\frac{1}{2}(n+1)^{1-\gamma}
-x^{1-\gamma})dx\right|\le \frac{1}{6}\gamma(1-\gamma)\frac{1}{n^{1+\gamma}}
$$
Using this result, we find $$
|\frac{3}{2}n^{1-\gamma}+\frac{1}{2}(n+2)^{1-\gamma}
+\frac{1}{2-\gamma}[(n+2)^{2-\gamma}-4(n+1)^{2-\gamma}+3n^{2-\gamma}]|
\le \frac{4}{6}\gamma(1-\gamma)\frac{1}{n^{1+\gamma}}+(n+2)^{1-\gamma}-(n+1)^{1-\gamma}
$$

Hence, the first two terms are controlled by $$
k^{-\gamma}\frac{1}{\Gamma(1-\gamma)}\frac{1+\gamma}{6(n+2)}\|u^2\|^2+k^{-\gamma}\frac{1}{2(n+2)}(\frac{1+\gamma}{\Gamma(1-\gamma)}+c_0/6)\|u^1\|^2
$$
The last term on right hand side is controlled by $$
k^{-\gamma}\frac{1}{n+2}(\frac{c_0}{2}+\frac{4}{6}\frac{1+\gamma}{\Gamma(1-\gamma)}+
\frac{(n+2)^{1-\gamma}-2^{1-\gamma}}{\Gamma(2-\gamma)})\|u^0\|^2
$$

Noting the signs of the coefficients, we have
\begin{multline*}
\frac{1}{2}k^{-\gamma}\left(c_0F_{n+1}+c_2F_{n-1}
+\sum_{m=3}^{n+1}c_mF_{n+1-m}\right)\le 
-\frac{|b_1|k^{-\gamma}}{8|n+2|}\|u^{n+1}\|^2+\\
\frac{k^{-\gamma}}{n+2}[\frac{\|u^2\|^2}{3\Gamma(1-\gamma)}+
(\frac{5}{8}|b_1|+\frac{1}{\Gamma(1-\gamma)}+\frac{c_0}{12})\|u_1\|^2
+(\frac{|b_1|}{8}+\frac{c_0}{2}+\frac{4}{3\Gamma(1-\gamma)}-\frac{2^{1-\gamma}}{\Gamma(2-\gamma)})\|u^0\|^2]\\
+\frac{1}{\Gamma(1-\gamma)(1-\gamma)}k^{-\gamma}(n+1)^{-\gamma}\|u^0\|^2
+2\lambda F_n.
\end{multline*}
Using Lemma \ref{lmm:stabilitylmm}, we can show that $F$ is bounded since $c=(b_0+\frac{3}{4}b_1,0, b_2+\frac{b_1}{4}, b_3,\ldots)$ satisfies the conditions in the lemma.
\end{proof}
\begin{rmk}
In the proof, we essentially use the sequence $c$ to show the stability. The bound obtained depends on $C_c$. As $\gamma\to 1$, $c_0\to 0$, and $C_c\to\infty$. Also, $\frac{1}{1-\gamma}$ blows up. The term $-k^{-\gamma}\|u^{n+1}\|^2$ in Equation .. becomes important to ensure the stability. One may combine these two to gain a uniform bound but we are not going to explore this.
\end{rmk}
The condition on $A_k$ says that $A_k$ should have a kind of negativity. For example, if $A_k(u)=-\delta u^{n+1}+B(u^{n-1}, u^n)$ where $\delta>0$ and $|B(u^{n-1}, u^n)|\le C(\|u^{n-1}\|+\|u^n\|)$, then the condition is satisfied. This theorem can be used to show that 
\begin{cor}
The implicit second order scheme for $D_c^{\gamma}u=\lambda u, \lambda\le 0$ is $l^2$-stable.
\end{cor}
This agrees with the $A$-stability analysis.
Actually, \tcb{as stated in ..., we have more. Now, we list the proof here...} 


Using the tools above, we can actually show some results about the first order scheme:

We denote the first order scheme as $$
(\mathscr{D}_ku)^{n+1}=k^{-\gamma}(a^{n+1}*u)_{n+1}.
$$

As we have seen, the difference scheme is $(b^{n+1}*u)_{n+1}$ where $b^{n+1}$ is a sequence with length $n+1$. 

\begin{thm}
If $E^n$ is a nonnegative sequence satisfying $$
(\mathscr{D}_kE)^{n+1}\le \lambda E^n
$$
then $$
E^n\le E^0 C(T)E_{\gamma}(C_1(nk)^{\gamma}), \forall 0\le n\le N.
$$
\end{thm}

\begin{proof}
Let $a=(a_0, a_1,\ldots)$ be the limit sequence.

As we know $a_0>0$ and $a_m<0$ for all $m\ge 1$. Then, we find that all components of $a^{(-1)}$ are positive.

By what has been computed, $a^{n+1}$ is exactly the same as the subsequence of $a$ for the first $n$ components. $$
k^{-\gamma}(a^{n+1}*E)_{n+1}=k^{-\gamma}(a*E)_{n+1}+k^{-\gamma}(a_{n+1}^{n+1}-a_{n+1})E^0\le CE^{n},\ \forall 0\le n\le N.
$$

Note that $a_{n+1}^{n+1}-a_{n+1}=\sum_{m=n+2}^{\infty}a_m=O(n^{-\gamma})$. \tcr{Can make exact using the explicit expressions}
\end{proof}

\begin{cor}
Consider the FODE $D_c^{\gamma}u=A(u)$. Suppose $\mathscr{A}_ku$ is a first order approximation for $A(u)$ such that $\exists\lambda\ge 0$, $$
\langle u^{n+1}, (\mathscr{A}_ku)^{n+1}\rangle\le \lambda(u^{n})^2.
$$
The scheme given by
$$
(\mathcal{D}_ku)^{n+1}=(\mathscr{A}_ku)^{n+1}, \ n\ge 1
$$
is stable.
\end{cor}

\begin{cor}
Consider the FODE $$
D_c^{\gamma}u=\lambda u.
$$
The first order explicit scheme given by $$
(\mathscr{D}_ku)^{n+1}=\lambda u^n,
$$
is stable for $\lambda\ge 0$.

The first order implicit scheme 
$$
(\mathscr{D}_ku)^{n+1}=\lambda u^{n+1},
$$
is stable for $\lambda\le 0$.
\end{cor}
\begin{proof}
Consider the explicit scheme and $\lambda\ge 0$. Since both the FODE and the scheme are linear. We can consider $u_0\ge 0$.
By the sign of the coefficients, we find that $u^n$ are all nonnegative.
The theorem then implies ....

For the implicit scheme, since $\langle u^{n+1}, \lambda u^{n+1}\rangle\le 0$. Using Corollary..., we find that ...
\end{proof}

\subsection{Analysis of fractional KS}

For the FKS, the continuous model,

Integrating by parts, $$
\frac{u(t)-u(0)}{t^{\gamma}}+\gamma\int_0^t\frac{u(t)-u(s)}{(t-s)^{1+\gamma}}ds
=\ldots
$$
It is clear that the difference scheme is a kind of approximation for this.
$b_0\approx \frac{1}{t^{\gamma}}+\gamma\int_0^{t-k}\frac{1}{(t-s)^{1+\gamma}}ds$, etc..

Since $$
\langle u, \Delta u-\nabla\cdot(\rho\nabla c)\rangle
=-\|\nabla u\|_2^2+\frac{1}{2}\|\rho^3\|_1,
$$
and $\|\rho^3\|_1\le C\|\rho\|_1\|\nabla\rho\|_2^2$ by Gargolidardo-Nirenberg inequality for dimension $2$, we see that if the initial mass is small, then, 
 $$
\langle u(t), \frac{u(t)-u(0)}{t^{\gamma}}+\gamma\int_0^t\frac{u(t)-u(s)}{(t-s)^{1+\gamma}}ds\rangle\le 0. 
$$
Since $a(a-b)\ge \frac{1}{2}(a^2-b^2)$, then $\|u(t)\|_2^2$ decays.


For the semi-discrete scheme, we just verify the conditions....



We just verify the conditions in....

\section{Numerical results}
\subsection{FODE}
stability region plot with variant $n$. \\

convergence order test for $D^\gamma_c u = \lambda u$ with BDF and variant BDF methods, where exact solution is available. \\


convergence order test for $D^\gamma_c u = f( u)$, where reference solution is computed with fine mesh.

\subsection{FKS}



\appendix

\section{Properties of the coefficients}

We now give the proof of Theorem ...
\begin{proof}
If we set $u=1$, then $u_s(s)-L_m'(s)=0$ for all $m\ge 1$. Hence, $$
k^{-\gamma}\sum_{m=0}^{n+1}b_m^{n+1}=\mathcal{D}_k1=D_c^{\gamma}1=0
$$
for any $n\ge 1$.

By the explicit formulas, $b_2^2=\frac{2^{1-\gamma}}{\Gamma(2-\gamma)}(\frac{2}{2-\gamma}-\frac{3}{2})$ and the inside is monotone which increases from $-1/2$ to $1/2$. For $n\ge 4$, the sign of $b_2^{n+1}=\frac{1}{\Gamma(2-\gamma)}h(2)$ is given by the discussion below. $b_2^4=\frac{1}{\Gamma(2-\gamma)}[h(2)-\frac{1}{2}3^{1-\gamma}-\frac{1}{2}4^{1-\gamma}-\frac{1}{2-\gamma}(3^{2-\gamma}-4^{2-\gamma})]$. The extra terms adding to $h(2)$ is nonpositive and is zero for $\gamma=0, \gamma=1$. By the property of $h(2)$ below, we find that $b_2^{4}$ is negative near $\gamma=0$ and positive near $\gamma=1$. Similarly, we find $\Gamma(2-\gamma)b_2^3=h(2)+\frac{3}{2}(2^{1-\gamma}+3^{1-\gamma})+\frac{3}{2-\gamma}(2^{2-\gamma}-3^{2-\gamma})$. The argument is similar as $b_2^4$.
The existence of $\bar{\gamma}_0$ and $\bar{\gamma}_1$ is proved.

The signs of other coefficients when $n=1, n=2$ can be checked directly since the explicit expressions are simple enough. The other claims about $n=1, n=2$ in the theorem can also be checked directly.

We now focus on $n\ge 3$. $b_0^{n+1}>0$ is obvious by the expressions of $b_0^{n+1}$. If $n\ge 2$, we have $$
\Gamma(2-\gamma)b_1^{n+1}=-1+\frac{2^{1-\gamma}}{2}+(-\frac{1}{2}-\frac{3}{2-\gamma}+\frac{2^{2-\gamma}}{2-\gamma})< 0.
$$

The existence of $b$ is obvious since $b_m^{n+1}$ is independent of $n$ when $n\ge 1$ and $m\le n-1$.

Since $\Gamma(2-\gamma)\ge 0$ and $\Gamma(2-\gamma)\to 1$ as $\gamma\to 1$ or $\gamma\to 0$, the properties of $b_m$ can be reduced to considering
\begin{multline*}
h(m)=-\frac{1}{2}(m-2)^{1-\gamma}-\frac{3}{2}m^{1-\gamma}+\frac{3}{2}(m-1)^{1-\gamma}+\frac{1}{2}(m+1)^{1-\gamma}\\
+\frac{1}{2-\gamma}(m+1)^{2-\gamma}
-\frac{3}{2-\gamma}m^{2-\gamma}
+\frac{3}{2-\gamma}(m-1)^{2-\gamma}
-\frac{1}{2-\gamma}(m-2)^{2-\gamma}.
\end{multline*}


Then, $$
h(2)=-\frac{3}{2}2^{1-\gamma}+\frac{3}{2}+\frac{1}{2}3^{1-\gamma}
+\frac{1}{2-\gamma}3^{2-\gamma}
+\frac{3}{2-\gamma}-\frac{3}{2-\gamma}2^{2-\gamma}.
$$
As $\gamma\to 0$, $h(2)\to 0$ and $h(2)\to 1/2$ as $\gamma\to 1$. 
Consider $H_2(\gamma)=(2-\gamma)h(2)=3(\gamma-6)2^{-\gamma}+(12-3\gamma/2)3^{-\gamma}
+(6-3\gamma/2)$. It's easy to see $H_2'(0)<0$. Further, 
\begin{multline*}
H_2''(\gamma)=3^{-\gamma}[3\ln 3+(12-3\gamma/2)(\ln 3)^2
-6(3/2)^{\gamma}\ln 2+(3/2)^{\gamma}(\gamma-6)(\ln 2)^2]\\
\ge 3^{-\gamma}[3\ln 3+(12-3\gamma/2)(\ln 3)^2
-9\ln 2+\frac{3}{2}(\gamma-6)(\ln 2)^2]
\end{multline*}
The inside is a linear function which is positive at both $\gamma=0$ and $\gamma=1$. Hence, $\exists \gamma_0>0$ such that $H_2(\gamma)<0$ when $\gamma\in(0,\gamma_0)$ and $H_2(\gamma)>0$ when $\gamma>\gamma_0$. Then, $h(2)$ has the same sign since $2-\gamma>0$.

For $m\ge 3$, we find $h(m)\to 0$ as $\gamma\to 0$ or $\gamma\to 1$. Setting $g=\frac{1}{2-\gamma}x^{2-\gamma}+\frac{1}{2}x^{1-\gamma}$, we have
$h(m)=-g(m-2)+3g(m-1)-3g(m)+g(m+1)$. $g'''<0$ for $x\ge 1$. Hence, $$
h(m)=\int_{m-2}^{m-1}g' dx-2\int_{m-1}^{m}g' dx
+\int_{m}^{m+1}g' dx<0,
$$
because $g'''<0$ implies that $g'$ is concave. $h(m)$ increases for $m\ge 4$ since $g''''>0$ for $x>3/2$. 
Direct computation shows that $b_3<b_4$.\tcr{Need to confirm.}
It is clear that $\lim_{m\to\infty}b_m=0$. By Taylor expansion about $m-1/2$, we find $$
h(m)=-(1-\gamma)\gamma m^{-1-\gamma}(1+O(1/m)),\ m\to\infty.
$$

\tcr{Now ,consider $n-1$, $n=3$ and $n=4$ are different, though the formula is uniform.}

Consider $h_1(n-1)$ where $n\ge 4$. \tcr{$h_1(2)$ corresponds to $b_2^{3+1}$ which has been discussed.}


\tcr{Then, $h_2(n)$ and $h_3(n+1)$}

The last four terms equal to $$
\frac{1}{2}n^{1-\gamma}+\frac{1}{2}(n+1)^{1-\gamma}
-\int_n^{n+1}x^{1-\gamma}dx<0
$$
because $x^{1-\gamma}$ is a concave function. 
If $n=2$, as $\gamma\to 1$, $\tilde{h}(n)\to 1/2$ and as $\gamma\to 0$, $\tilde{h}(n)\to 0$. Hence, it is clear that $\tilde{h}(2)$ is negative if $\gamma$ is close to $0$ and positive if $\gamma$ is close to $1$. 
If $n\ge 3$, $\tilde{h}(n)<0$. By Taylor expansion about $n+1/2$, the last four terms add to $O(n^{-1-\gamma})$. It is also clear that when $n\ge 3$, as $\gamma\to 0,1$, $\tilde{h}(n)\to 0$. 

Now, consider $\bar{h}(n+1)$ for $n\ge 2$ where 
\begin{multline*}
\bar{h}(n+1)
=-(n-1)^{1-\gamma}\frac{1}{2}+\frac{1}{2}n^{1-\gamma}
-\frac{1}{2-\gamma}(n-1)^{2-\gamma}+\frac{1}{2-\gamma}n^{2-\gamma}
-(n+1)^{1-\gamma}.
\end{multline*}
As $\gamma\to 1$, $\bar{h}(n+1)\to 0$ and as $\gamma\to 1$, $\bar{h}(n+1)\to -1$. 

We find 
\begin{multline*}
\bar{h}(n+1)=\int_{n-1}^nx^{1-\gamma}dx-(n+1)^{1-\gamma}+\frac{1}{2}n^{1-\gamma}
-\frac{1}{2}(n-1)^{1-\gamma}\\
<-(n+1)^{1-\gamma}+\frac{3}{2}n^{1-\gamma}-\frac{1}{2}(n-1)^{1-\gamma}<0
\end{multline*}
Further, as $n\to\infty$, by Taylor expansion about $n$, we find that $\bar{h}(n+1)=O(n^{-\gamma})$.

Now, we compute directly 
\begin{gather*}
\Gamma(2-\gamma)(b_0+\frac{3}{4}b_1)
=-\frac{5}{8}-\frac{5}{4}\frac{1}{2-\gamma}+\frac{3}{4}2^{-\gamma}
+\frac{3}{2-\gamma}2^{-\gamma}=:g_1(\gamma)\\
\Gamma(2-\gamma)(|b_1|-4b_2)=-\frac{9}{2}-\frac{8-\gamma}{2-\gamma}6\cdot3^{-\gamma}+\frac{6-\gamma}{2-\gamma}11\cdot2^{-\gamma}-\frac{9}{2-\gamma}=:g_2(\gamma)
\end{gather*}
$g_1=\frac{3}{2-\gamma}(2^{-\gamma}-\frac{1}{2})+
\frac{1}{4(2-\gamma)}+\frac{3}{4}2^{-\gamma}-\frac{5}{8}>\frac{1}{4(2-\gamma)}+\frac{3}{4}2^{-\gamma}-\frac{5}{8}:=\tilde{g}_1$. $\tilde{g}_1$ is a convex function and $\tilde{g}_1(1)=0$, $\tilde{g}_1'(1)<0$, then $\tilde{g}_1>0$. 
$g_2(0)=g_2(1)=0$. Note that $[(2-\gamma)g_2]''
=3^{-\gamma}[-12\ln 3-6(8-\gamma)\ln(3)^2+22\cdot 1.5^{\gamma}\ln 2+11(6-\gamma)1.5^{\gamma}(\ln 2)^2]$. Clearly, $-12\ln 3-6(8-\gamma)\ln(3)^2+22\cdot 1.5^{\gamma}\ln 2+11(6-\gamma)1.5^{\gamma}(\ln 2)^2<-12\ln 3-6(8-\gamma)\ln(3)^2+33\ln 2+16.5(6-\gamma)(\ln 2)^2$. The right hand side is a linear function and the values of this linear function at two endpoints are negative. Hence, $[(2-\gamma)g_2]''<0$. Hence, $(2-\gamma)g_2$ is concave and $g_2>0$.

Then, the third part of the theorem and the second part of the theorem 
for $n\ge 2$ are proved.
\end{proof}

\section{The negativity of $P_m$}
\begin{lmm}
$P_m(\tau)\le 0$ for all $m\ge 1$ and $\tau\in I_m$.
\end{lmm}
\begin{proof}
\begin{multline*}
P_m(\tau)=(t^{m-1}-\tau)^2k^{2-\gamma}[\frac{1}{2}(n-m)^{1-\gamma}+\frac{1}{2-\gamma}(n-m)^{2-\gamma}]+\\
(-2)(t^m-\tau)^2k^{2-\gamma}[(n-m)^{1-\gamma}+\frac{1}{2-\gamma}(n-m)^{2-\gamma}]\\
+(t^{m+1}-\tau)^2k^{2-\gamma}[\frac{1}{2}(n-m+1)^{1-\gamma}+\frac{1}{2-\gamma}(n-m+1)^{2-\gamma}]\\
-\frac{2k^2}{2-\gamma}(t^{n+1}-\tau)^{2-\gamma}
\end{multline*}

$P_m(\tau=t^{m+1})=0$ and
$P_m(\tau=t^m)=k^2\int_{I_m}\frac{t^{m+1/2}-s}{(t^{n+1}-s)^{\gamma}}ds<0$.

We find that 
\begin{multline*}
(1-\gamma)P'=2(\tau-t^{m-1})k^{2-\gamma}[\frac{1}{2}(n-m)^{1-\gamma}+\frac{1}{2-\gamma}(n-m)^{2-\gamma}]+\\
(-2)2(\tau-t^m)k^{2-\gamma}[(n-m)^{1-\gamma}+\frac{1}{2-\gamma}(n-m)^{2-\gamma}]\\
+2(\tau-t^{m+1})k^{2-\gamma}[\frac{1}{2}(n-m+1)^{1-\gamma}+\frac{1}{2-\gamma}(n-m+1)^{2-\gamma}]\\
+2k^2(t^{n+1}-\tau)^{1-\gamma}
\end{multline*}
We can show that $P'(t^m)<0$. $P'(t^{m+1})=0$. 

Further, 
\begin{multline*}
(1-\gamma)P''(t^{m+1})/k^{2-\gamma}=
-3(n-m)^{1-\gamma}-\frac{2}{2-\gamma}(n-m)^{2-\gamma}+\\
+[(n-m+1)^{1-\gamma}+\frac{2}{2-\gamma}(n-m+1)^{2-\gamma}]
-2(1-\gamma)(n-m)^{-\gamma}\\
=(n-m+1)^{1-\gamma}-3(n-m)^{1-\gamma}
+2\int_{n-m}^{n-m+1}x^{1-\gamma}dx-2(1-\gamma)(n-m)^{-\gamma}
\end{multline*}
Using the inequality $x^{1-\gamma}\le (n-m)^{1-\gamma}+(1-\gamma)(n-m)^{-\gamma}(x-(n-m))$ since the function is concave, we find $$
(1-\gamma)P''(t^{m+1})/k^{2-\gamma}\le (n-m+1)^{1-\gamma}-(n-m)^{1-\gamma}
-(1-\gamma)(n-m)^{-\gamma}\le 0.
$$
Hence, $P''$ could be positive on $\tau\in (t^m, t_0)$ and negative on $(t_0, t^{m+1})$ or all negative on $(t^m, t^{m+1})$. Together with the fact $P'(t^{m+1})=0$, we know either $P$ either first decreases and then increases or increases for all time. Since $P(t^{m+1})=0$ and $P(t^m)<0$, we find that $P\le 0,\tau\in (t^m, t^{m+1})$.
\end{proof}

\bibliographystyle{unsrt}
\bibliography{frac}
\end{document}
