% !TEX TS-program = pdflatex
% !TEX encoding = UTF-8 Unicode

% This is a simple template for a LaTeX document using the "article" class.
% See "book", "report", "letter" for other types of document.

\documentclass[11pt]{article} % use larger type; default would be 10pt

\usepackage[utf8]{inputenc} 
\usepackage{geometry} % to change the page dimensions
\geometry{a4paper} 

\usepackage{mathrsfs, graphics,amssymb,amsmath,subeqnarray,setspace,graphicx,amsthm,epstopdf,subfig,color}
\usepackage{booktabs,array,paralist,verbatim} 
\usepackage{enumitem}

%%% HEADERS & FOOTERS
\usepackage{fancyhdr} % This should be set AFTER setting up the page geometry
\pagestyle{fancy} % options: empty , plain , fancy
\renewcommand{\headrulewidth}{0pt} % customise the layout...
\lhead{}\chead{}\rhead{}
\lfoot{}\cfoot{\thepage}\rfoot{}

%%% SECTION TITLE APPEARANCE
\usepackage{sectsty}
\allsectionsfont{\sffamily\mdseries\upshape} % (See the fntguide.pdf for font help)
% (This matches ConTeXt defaults)

\usepackage[nottoc,notlof,notlot]{tocbibind} 
\usepackage[titles,subfigure]{tocloft} 

\setlist[enumerate]{leftmargin=*}

\renewcommand{\cftsecfont}{\rmfamily\mdseries\upshape}
\renewcommand{\cftsecpagefont}{\rmfamily\mdseries\upshape} % No bold!
\newcommand{\tcb}[1]{\textcolor{blue}{#1}}
\newcommand{\tcr}[1]{\textcolor{red}{#1}}
\newcommand{\zz}[1]{\textcolor{blue}{ZZ: #1}}

\newtheorem{thm}{Theorem}
\newtheorem{lmm}{Lemma}
\newtheorem{rmk}{Remark}
\newtheorem{pro}{Proposition}
\newtheorem{cor}{Corollary}
%%% END Article customizations

%%% The "real" document content comes below...

\title{Second order BDF method for FODE and FKS equations}
\author{The Authors}
%\date{} % Activate to display a given date or no date (if empty),
         % otherwise the current date is printed 

\begin{document}
\maketitle

\section{Introduction}

\section{The schemes and main results}


For $0<\gamma\le 1$, consider the Caputo derivative defined as $$
D_c^{\gamma}u=\frac{1}{\Gamma(1-\gamma)}\int_0^t\frac{u'(s)}{(t-s)^{\gamma}}ds.
$$
We aim to design a second backward difference formula for the fraction differential equation (abbreviated by FDE henceforth)
\[
D_c^{\gamma}u =f(u,t).
\]  
Here, if $f(u,t)$ is a pure function of $u$ and $t$, this equation is a fractional ordinary differential equation (abbreviated by FODE). And when $f(u,t)$ only depends on spatial derivatives of $u$, this equation is a  fractional partial differential equation 
(abbreviated by FPDE). The Caputo derivative is introduced in the place of the regular derivative to model the memory effect. For stiff FODE's and FPDE's, the BDF approximation to this model is desired for stability reason. 


\subsection{Numerical approximation of the Caputo derivative}
\subsubsection{Review of the first order scheme and other existing results}

\subsubsection{The second order BDF scheme}

In order to for a clear unambiguous presentation of the scheme, we introduce the following notations.   Let $a\in \mathbb{R}^{\mathbb{N}}$ be a sequence of real numbers, $a=(a_0, a_1, \ldots, a_m, \ldots)$ .
Its generating function is given by $$
F_a(z)=\sum_{m=0}^{\infty}a_m z^m.
$$
For $a, c\in \mathbb{R}^{\mathbb{N}}$, we define the convolution $a*c$ as $$
(a*c)_n=\sum_{m=0}^{n}a_m c_{n-m},
$$
and clearly $F_{a*c}(z)=F_a(z)F_c(z)$. $\delta_d=(1,0,\ldots)$ is the convolution identity,  whose generating function is $1$. This means, 
\[
a * \delta_d =a, \quad \forall a \in \mathbb{R}^{\mathbb{N}}.
\] 
For $b\in \mathbb{R}^{\mathbb{N}}$ with $b_0\neq 0$, we denote by $b^{(-1)}$ its convolution inverse, meaning that
\[
b^{(-1)} * b = b *b^{(-1)} =\delta_d.
\]  
Clearly, for a sequence $b$ ($b_0\neq 0$) with generating function $F(z)$, the generation function of the convolution inverse of $b^{(-1)}$ is given by $1/F(z)$.

We introduce the following partition in time. Consider a time step $k$ and $t^n=nk$. We assume that $T=Nk$ is a fixed constant. 
Let $u=(u_0, u_1, u_2, \ldots)\in \mathbb{R}^{\mathbb{N}}$ be the sequence of nodal values of $u$. \zz{better distinguish u as a continuous function and as a sequence of discrete data...}  We propose a second order backward differentiation scheme for $D_c^{\gamma}u$ below, which is a  linear multistep method. For simplicity, we will assume that $u_0$ and $u_1$ are given. 

For $n\ge 1$, the derivative is approximated by $$
D_c^{\gamma}u(t^{n+1})\approx
k^{-\gamma}(b^{n+1}*u)_{n+1}=:\mathcal{D}_k u.
$$
Here $b^{n+1}=(b_0^{n+1},\cdots,b_{n+1}^{n+1})$ is a sequence with $n+2$  non-zero components. Actually, we can extend the finite sequence $b^{n+1}$  by setting $b_m^{n+1}=0, \,m\ge n+2$, then with a bit abuse of notation, we still denote $b^{n+1}\in\mathbb{R}^{\mathbb{N}}$. Therefore $b^{n+1}*u$ is interpreted according to the definition above, and 
\[
(b^{n+1}*u)_{n+1}=\sum_{m=0}^{n+1}b_{m}^{n+1}u_{n+1-m}
\]
 is the $(n+1)$-th component of this convolution.

The family of sequences $b^{n+1}$ are given as follows: (obviously, it suffices to define the first $n+2$ components of each sequence)

For $n=1$,
\begin{gather*}
\Gamma(2-\gamma)b_{0}^{2}=-\frac{1}{2}2^{1-\gamma}+\frac{1}{2-\gamma}2^{2-\gamma},\\
\Gamma(2-\gamma)b_{1}^{2}=2\cdot 2^{1-\gamma}-\frac{2}{2-\gamma}2^{2-\gamma},\\
\Gamma(2-\gamma)b_{2}^{2}=-\frac{3}{2}2^{1-\gamma}+\frac{1}{2-\gamma}2^{2-\gamma}.
\end{gather*}

For $n=2$,
\begin{gather*}
\Gamma(2-\gamma)b_{0}^{3}=\frac{1}{2}+\frac{1}{2-\gamma},\\
\Gamma(2-\gamma)b_{1}^{3}
=-\frac{3}{2}-\frac{3}{2-\gamma}-\frac{1}{2}3^{1-\gamma}
+\frac{1}{2-\gamma}3^{2-\gamma},\\
\Gamma(2-\gamma)b_{2}^{3}=\frac{3}{2}+2\cdot 3^{1-\gamma}
+\frac{3}{2-\gamma}-\frac{2}{2-\gamma}3^{2-\gamma},\\
\Gamma(2-\gamma)b_{3}^{3}=-\frac{1}{2}
-\frac{3}{2}3^{1-\gamma}+\frac{1}{2-\gamma}(3^{2-\gamma}-1).
\end{gather*}

For $n\ge 3$:

\begin{gather}
\begin{array}{c}
\Gamma(2-\gamma)b_{0}^{n+1}=\frac{1}{2}+\frac{1}{2-\gamma},\\
\Gamma(2-\gamma)b_1^{n+1}=-\frac{3}{2}+2^{1-\gamma}\frac{1}{2}
-\frac{3}{2-\gamma}+\frac{1}{2-\gamma}2^{2-\gamma}.
 \end{array}
\end{gather}
And when $2\le m\le n-2$, we have $\Gamma(2-\gamma)b_{m}^{n+1}=h(m)$ where the function $h$ is given by
\begin{multline}
h(m)=-\frac{1}{2}(m-2)^{1-\gamma}-\frac{3}{2}m^{1-\gamma}+\frac{3}{2}(m-1)^{1-\gamma}+\frac{1}{2}(m+1)^{1-\gamma}\\
+\frac{1}{2-\gamma}(m+1)^{2-\gamma}
-\frac{3}{2-\gamma}m^{2-\gamma}
+\frac{3}{2-\gamma}(m-1)^{2-\gamma}
-\frac{1}{2-\gamma}(m-2)^{2-\gamma}.
\end{multline}
For the last three coefficients,
\begin{gather}
\Gamma(2-\gamma)b_{n-1}^{n+1}
=-\frac{1}{2}n^{1-\gamma}-\frac{1}{2}(n+1)^{1-\gamma}
-\frac{1}{2-\gamma}[n^{2-\gamma}-(n+1)^{2-\gamma}]
+h(n-1),\\
\Gamma(2-\gamma)b_n^{n+1}=h(n)+\frac{3}{2}n^{1-\gamma}+\frac{3}{2}(n+1)^{1-\gamma}
+\frac{3}{2-\gamma}[n^{2-\gamma}-(n+1)^{2-\gamma}],\\
\Gamma(2-\gamma)b_{n+1}^{n+1}
=-\frac{1}{2}(n-1)^{1-\gamma}-\frac{3}{2}(n+1)^{1-\gamma}
-\frac{1}{2-\gamma}(n-1)^{2-\gamma}+\frac{1}{2-\gamma}(n+1)^{2-\gamma}.
\end{gather}

\zz{I moved the remark to Chapter 3, not a good fit here.}


\subsection{Main results on the second order BDF method for FODE}
stability and accuracy results of the BDF method for $D^\gamma_c u = \lambda u$. \\
stability results on variant BDF method.

\begin{thm}
Suppose $u\in C^3([0,\infty), \mathbb{R})$. Then, $\exists C>0$ independent of $\gamma$ and $n$ such that $\forall 1\le  n\le N-1$, $$
|D_c^{\gamma}u(t^{n+1})-k^{-\gamma}(b^{n+1}*u)_{n+1}|\le Ck^{3-\gamma}.
$$
\end{thm}


As we shall see in Theorem .. below, there exists $b\in\mathbb{R}^{\mathbb{N}}$, such that $b^{n+1}_m=b_m$ for $m\le n-2$, and $b^{n+1}\to b$ in $l^p, p\ge 1$. \tcr{Motivation of the new definition of $A$-stability and the new stability region.}
The scheme is called $A$-stable if the stability region contains the whole left half plane. Clearly, the $A$-stability is reduced to study of the generating function of $b$.
\begin{thm}
The second backward differentiation $\mathcal{D}_k$ is $A$-stable. 
\end{thm}

\begin{thm}
Consider the FODE $D_c^{\gamma}u=A(u)$. Suppose $A_ku$ is a second order approximation for $A(u)$ such that $\exists\lambda\ge 0$, $$
\langle u^{n+1}, (A_ku)^{n+1}\rangle \le \lambda(\|u^{n}\|^2+\|u^{n-1}\|^2).
$$
The scheme given by
$$
(\mathcal{D}_ku)^{n+1}=(A_ku)^{n+1}, \ n\ge 1
$$
is $l^2$-stable, or in other words, $$
\sum_{m=0}^n |u_m|^2k\le C(T)E_{\gamma}(C_1(nk)^{\gamma})
$$
for all $n\le N$, where $E_{\gamma}$ is the Mittag-Leffler function.\tcr{Need to make precise the constants}
\end{thm}



\subsection{Main results on the first/second order method for FKS}
stability.\\
small data condition?

Consider the following fractional Keller-Segel equation:
\begin{gather*}
D_c^{\gamma}\rho=\Delta\rho-\nabla\cdot(\rho\nabla c)\\
-\Delta c=\rho
\end{gather*}

\begin{pro}
The first order scheme for the KS is stable with small initial data.
\end{pro}


\begin{pro}
The second order scheme for the KS is $l^2(L^2(\mathbb{R}^2))$-stable if the initial data is small.
\end{pro}


By the sign properties of of $b^{n+1}$, we have 
\begin{pro}
For the FKS and the second order scheme with five point.. \tcr{to complete},
$\exists \bar{\gamma}_0\in (0, 1)$, if $\gamma\le\bar{\gamma}_0$ and $u_0\ge 0, u_1\ge 0$, then $u_m\ge 0$ for all $m\ge 0$.
\end{pro}

\section{Construction of the numerical approximation of the Caputo derivative}
derivation.  \\
estimates on the coefficients.

To be convenient, we denote $I_m=[t^m, t^{m+1}]$.

Suppose $u_m$ is the nodal value of $u$ at $t^m$. The Lagrange interpolation on $I_m$ is given by:
$$
L_m(t)=u_{m-1}\frac{(t-t^m)(t-t^{m+1})}{2k^2}-u_m\frac{(t-t^{m-1})(t-t^{m+1})}{k^2}+u_{m+1}\frac{(t-t^m)(t-t^{m-1})}{2k^2}
$$
Hence, we use the approximation for $m\ge 1$
\begin{gather}
u'(t)\approx L_m'(t)=u_{m-1}\frac{t-t^{m+1/2}}{k^2}-2u_m\frac{t-t^m}{k^2}+u_{m+1}\frac{t-t^{m-1/2}}{k^2},\ t\in I_m.
\end{gather}
For $t\in I_0$, $L_0$ will be the quadratic approximation using $u_0, u_1, u_2$, i.e., we set $L_0=L_1$.

The numerical scheme is given by
\begin{gather}
\mathcal{D}_ku(t^{n+1})\approx\frac{1}{\Gamma(1-\gamma)}\sum_{m=0}^n
\int_{I_m}\frac{L_m'(s)}{(t^{n+1}-s)^{\gamma}}ds=k^{-\gamma}\sum_{m=0}^{n+1}b^{n+1}_m u_{n+1-m}.
\end{gather}
We call this formula `backward differentiation' since $b_0^{n+1}>0$. As we shall see the scheme tends to BDF2 as $\gamma\to 1$.

We define the coefficients $C_m^p$ for $m\ge 1$ through: $$
k^{\gamma}(1-\gamma)\int_{I_m}\frac{L_m'(s)}{(t^{n+1}-s)^{\gamma}}ds=C_m^{m+1}u_{m+1}+C_m^mu_m+C_{m}^{m-1}u_{m-1}.
$$
Hence,
\begin{gather*}
\begin{array}{cl}
C_m^{m+1}=&-\frac{3}{2}(n-m)^{1-\gamma}+\frac{1}{2}(n-m+1)^{1-\gamma}
-\frac{1}{2-\gamma}(n-m)^{2-\gamma}+\frac{1}{2-\gamma}(n-m+1)^{2-\gamma},\\
C_m^m=&2(n-m)^{1-\gamma}+\frac{2}{2-\gamma}(n-m)^{2-\gamma}
-\frac{2}{2-\gamma}(n-m+1)^{2-\gamma},\\
C_m^{m-1}=&-\frac{1}{2}(n-m)^{1-\gamma}-\frac{1}{2}(n-m+1)^{1-\gamma}
-\frac{1}{2-\gamma}(n-m)^{2-\gamma}+\frac{1}{2-\gamma}(n-m+1)^{2-\gamma}.
\end{array}
\end{gather*}
These are only valid for $m\ge 1$.

For $m=0$, we compute $k^{\gamma}(1-\gamma)\int_0^{t^1}\frac{L_1'(t)}{(t^{n+1}-s)^{\gamma}}ds$ where
\begin{gather*}
L_1'(t)=u_{0}\frac{t-t^{3/2}}{k^2}-2u_1\frac{t-t^1}{k^2}+u_{2}\frac{t-t^{1/2}}{k^2}.
\end{gather*}
and find
\begin{gather*}
C_0^2=-\frac{1}{2}n^{1-\gamma}-\frac{1}{2}(n+1)^{1-\gamma}
-\frac{1}{2-\gamma}[n^{2-\gamma}-(n+1)^{2-\gamma}],, \\
C_0^1=2(n+1)^{1-\gamma}
+\frac{2}{2-\gamma}[n^{2-\gamma}-(n+1)^{2-\gamma}], \\
C_0^0=\frac{1}{2}n^{1-\gamma}-\frac{3}{2}(n+1)^{1-\gamma}
-\frac{1}{2-\gamma}[n^{2-\gamma}-(n+1)^{2-\gamma}]
\end{gather*}
Hence, by the definition, we have
\begin{multline*}
D_c^{\gamma}u(t^{n+1})\approx \frac{1}{k^{\gamma}}\sum_{m=0}^{n+1}b_{m}^{n+1}u_{n+1-m}=\\
\frac{1}{k^{\gamma}\Gamma(2-\gamma)}(C_0^0u_0+C_0^1u_1+C_0^2u_2+\sum_{m=1}^n(C_m^{m-1}u_{m-1}
+C_m^{m}u_m+C_m^{m+1}u_{m+1}))
\end{multline*}
Using the explicit formulas, we obtain the coefficients listed in Section...


\begin{thm}
We have the following claims:
\begin{enumerate}
\item 
For all $n\ge 1$, $$
\sum_{m=0}^{n+1}b_m^{n+1}=0.
$$ 
For all $n\ge 1$, $b_0^{n+1}>0$, $b_1^{n+1}<0$. For $n\ge 2$, $b_m^{n+1}<0$, $m\ge 3$. $\exists \bar{\gamma}_0\in (0, 1)$ and $\bar{\gamma}_1\in(\bar{\gamma}_0, 1)$ such that $b_2^{n+1}\le 0$ for $n\ge 1, \gamma<\bar{\gamma}_0$ and $b_2^{n+1}>0$ for $n\ge 1, \gamma>\bar{\gamma}_1$.


There exists a sequence $b\in \mathbb{R}^{\mathbb{N}}$ such that if $n\ge 2$, $b_m^{n+1}=b_m$ for $m\le n-2$.

\item 
When $n\ge 1$, as $\gamma\to 1$, $$
b_0^{n+1}\to 3/2, b_1^{n+1}\to -2, b_2^{n+1}\to 1/2, b_m^{n+1}\to 0
$$
and the scheme tends to BDF2; as $\gamma\to 0$, $$
b_0^{n+1}\to 1, b_m^{n+1}\to 0 ,b_{n+1}^{n+1}\to -1.
$$
\item As $n\to \infty$, $b_{n-1}^{n+1}=O(n^{-1-\gamma})$ ,$b_n^{n+1}=O(n^{-1-\gamma})$ and $b_{n+1}^{n+1}=O(n^{-\gamma})$. $b^{n+1}\to b$ pointwise and in $l^p$, $p\ge 1$. 

For the limiting sequence $b$, $b_0>0$, $b_m<0$
for $m\ge 1, m\neq 2$, and $$
\sum_m b_m=0,\ 
b_0\ge -\frac{3}{4}b_1,\ \ -b_1\ge 4b_2.
$$
$\exists\gamma_0\in (0,1)$ such that $b_2\le 0$ if $\gamma\le \gamma_0$ and $b_2>0$ for $\gamma>\gamma_0$. 
When $m\ge 3$, $b_m$ increases to zero and $b_m=\frac{1}{\Gamma(-\gamma)}\frac{1}{m^{1+\gamma}}\left(1+O(\frac{1}{m})\right)$ as $m\to\infty$.
\end{enumerate}
\end{thm}

\begin{rmk}
In simulation, we can compute $u^1$, for example using a first order scheme. The one step error for $u^1$ will then be $O(k^2)$ \tcr{Need to confirm.}. This is not as good as the $O(k^{3-\gamma})$ estimation below. An alternative way is to $L_1(t)$ on both $I_0, I_1$. The relations 
\begin{gather*}
D_c^{\gamma}u(t^1)\approx k^{-\gamma}\frac{1}{\Gamma(2-\gamma)}[(-\frac{1}{2}
+\frac{1}{2-\gamma})u_2+(2-\frac{2}{2-\gamma})u_1+(-\frac{3}{2}+\frac{1}{2-\gamma})u_0],\\
D_c^{\gamma}u(t^2)\approx k^{-\gamma}b^{2}*u
\end{gather*}
then form a system of equations for $u^1, u^2$. We then compute them all at once. The local truncation error is $O(k^{3-\gamma})$.
\end{rmk}


\section{Stability and convergence analysis}


\subsection{Local truncation error}
Suppose $u$ is $C^3$. We know estimate the local truncation error.

Proof of Theorem:

By the Taylor formula $$
f(t)=f(s)+(t-s)f'(s)+\frac{1}{2}(t-s)^2f''(s)+\int_s^t\frac{1}{2}(t-\tau)^2 f'''(\tau)d\tau,
$$
we find 
\begin{multline}
R_m(s)=L_m'(s)-u_s(s)=\frac{s-t^{m+1/2}}{k^2}\int_s^{t^{m-1}}\frac{1}{2}(t^{m-1}-\tau)^2 u'''(\tau)d\tau\\
-2\frac{s-t^m}{k^2}\int_s^{t^m}\frac{1}{2}(t^m-\tau)^2 u'''(\tau)d\tau
+\frac{s-t^{m-1/2}}{k^2}\int_s^{t^{m+1}}\frac{1}{2}(t^{m+1}-\tau)^2 u'''(\tau)d\tau
\end{multline}
The total error will be $$
r=\frac{1}{\Gamma(1-\gamma)}\left(r_0+\sum_{m=1}^n\int_{I_m}\frac{R_m(s)}{(t^{n+1}-s)^{\gamma}}ds\right)
$$
where $$
r_0=\int_{I_0}\frac{R_1(s)}{(t^{n+1}-s)^{\gamma}}ds.
$$

Direct estimate shows $$
r_0\le Ck^2\int_0^k\frac{1}{((n+1)k-s)^{\gamma}}ds\le C_1k^{3-\gamma}.
$$
where $C_1$ is independent of $n,\gamma$. If we use the same estimate, $|R_m|\le Ck^2$, then we obtain $$
r\le \frac{1}{\Gamma(1-\gamma)}(r_0+Ck^2\int_{t^1}^{t^{n+1}}\frac{1}{(t^{n+1}-s)^{\gamma}}ds)\le \frac{C}{\Gamma(2-\gamma)}T^{1-\gamma}k^2
$$
The scheme is at least uniformly second order.  

Better estimates 
\begin{gather*}
\frac{1}{\Gamma(1-\gamma)}\int_{t^m}^{t^{m+1}}\frac{R_m(s)}{(t^{n+1}-s)^{\gamma}}ds=I_1^m+I_2^m
\end{gather*}
where $$
I_1^m=\frac{1}{\Gamma(1-\gamma)}\frac{1}{2k^2}\int_{t^{m-1}}^{t^{m}}
u'''(\tau)(t^{m+1}-\tau)^2\int_{t^m}^{t^{m+1}}\frac{t^{m+1/2}-s}{(t^{n+1}-s)^{\gamma}}dsd\tau,
$$
and $I_2^m=\frac{1}{2k^2\Gamma(1-\gamma)}\int_{t^m}^{t^{m+1}}u'''(\tau)P_m(\tau)d\tau$,
\begin{multline*}
P_m(\tau)=(t^{m-1}-\tau)^2\int_{\tau}^{t^{m+1}}\frac{t^{m+1/2}-s}{(t^{n+1}-s)^{\gamma}}ds\\
-2(t^m-\tau)^2\int_{\tau}^{t^{m+1}}\frac{t^{m}-s}{(t^{n+1}-s)^{\gamma}}ds
+(t^{m+1}-\tau)^2\int_{t^m}^{\tau}\frac{s-t^{m-1/2}}{(t^{n+1}-s)^{\gamma}}ds.
\end{multline*}

Note that $\int_{t^m}^{t^{m+1}}\frac{t^{m+1/2}-s}{(t^{n+1}-s)^{\gamma}}ds$
is negative for all $m$, we find $$
|\sum_m I_1^m|\le \sum_m \frac{Ck}{\Gamma(1-\gamma)}\int_{t^m}^{t^{m+1}}\frac{s-t^{m+1/2}}{(t^{n+1}-s)^{\gamma}}ds
$$
where $C\sim \sup|u'''|$ is independent of $n,m,\gamma$.
\begin{multline*}
\int_{t^m}^{t^{m+1}}\frac{s-t^{m+1/2}}{(t^{n+1}-s)^{\gamma}}ds
=\frac{k^{2-\gamma}}{1-\gamma}[
\frac{1}{2-\gamma}((n-m+1)^{2-\gamma}-(n-m)^{2-\gamma})
-\frac{1}{2}((n-m+1)^{1-\gamma}+(n-m)^{1-\gamma})]
\end{multline*}
This number $\le$ $\frac{Ck^{2-\gamma}}{1-\gamma}(n-m+1/2)^{-1-\gamma}$ by Taylor expansion about $n-m+1/2$, where $C=\sup_{\gamma}\gamma(1-\gamma)$ is independent of $m,\gamma, n$. Hence, $$
|\sum_{m\ge 1} I_1^m|\le \sum_{m=1}^n
\frac{1}{\Gamma(1-\gamma)}\frac{Ck^{3-\gamma}}{1-\gamma}(n-m+1/2)^{-1-\gamma}\le C_1k^{3-\gamma}
$$
where $C_1$ is independent of $\gamma$ and $n$.

For $I_2^m$, in the appendix, we show that $P_m(\tau)\le 0$ for all $m\ge 1$ and $\tau\in I_m$. Applying the integral mean value theorem, we obtain  $\int_{t^m}^{t^{m+1}}u'''(\tau)P_m(\tau)d\tau
=u'''(\xi) \int_{I_m}P_m(\tau)d\tau$. Direct computation shows
\begin{multline*}
(1-\gamma)\int P_m(\tau)d\tau=\frac{k^3}{2}k^{2-\gamma}(n-m)^{1-\gamma}+
\frac{5}{3}\frac{1}{2-\gamma}(n-m)^{2-\gamma}\\
+(\frac{k^3}{3})k^{2-\gamma}[\frac{1}{2}(n-m+1)^{1-\gamma}+\frac{1}{2-\gamma}(n-m+1)^{2-\gamma}]\\
+\frac{2k^2}{(2-\gamma)(3-\gamma)}(n-m)^{3-\gamma}k^{3-\gamma}
-\frac{2k^2}{(2-\gamma)(3-\gamma)}(n-m+1)^{3-\gamma}k^{3-\gamma}
\end{multline*}
We do Taylor expansion about $n-m+1/2$. We find $2-\gamma$, $1-\gamma$, $-\gamma$ all cancel out. 
The nonzero power will be $(n-m+1/2)^{-1-\gamma}$, and hence summable.

Hence,  $$
|\sum_{m\ge 1}I_2^m|\le\sum_{1\le m\le n}\frac{1}{k^2\Gamma(1-\gamma)} \frac{C_2}{1-\gamma}k^{5-\gamma}(n-m+1/2)^{-1-\gamma}
$$
Hence, $$
r\le \frac{1}{\Gamma(1-\gamma)}r_0+C_3k^{3-\gamma}\le Ck^{3-\gamma},
$$
where $C$ is uniform for $\gamma\in [0,1]$ since $\Gamma(1-\gamma)\ge 1$.


\subsection{A-stability}
 In the case $b_2\le 0$, the scheme is $A$-stable. This is because all coefficients except $b_0$ are all negative. $|b_0-z|\le \sum_{m=1}^{\infty}(-b_m^{\infty})|\eta|^m\le b_0$
if $|\eta|\le 1$. 

Now, assume $b_2>0$. It would suffice to show $\inf_{z\in D(0,1)}Re(f(\eta, 0))\ge 0$. Recall that $b_1<0, b_2>0$

Since $-|\sum_{m=3}^{\infty}b_m\eta^m|\ge -\sum_{m=3}^{\infty}|b_m|
=\sum_{m=3}^{\infty}b_m=-(b_0+b_1+b_2)$, we find 
 $Re(f(\eta,0))\ge b_0+Re(b_1\eta+b_2\eta^2)-(b_0+b_1+b_2)$

Setting $\eta=r\cos\theta$ results in the function $$
h(r,\theta)=|b_1|(1-r\cos\theta)-b_2(1-r^2\cos(2\theta)).
$$
This function is positive for $\cos\theta\le 0$ because $|b_1|\ge 2b_2$. For $\cos\theta\ge 0$, $h_r=-|b_1|\cos\theta+2b_2r\cos(2\theta)\le 0$ since $r\cos(2\theta)\le\cos(\theta)$. The minimum value is achieved on $r=1$. 

Set $h(\theta)=|b_1|(1-\cos\theta)-b_2(1-\cos(2\theta))$ with $0\le\theta\le \pi/2$. We find that $h(0)=0$, which is desired since $f(1,0)=0$. $h'=\sin\theta (|b_1|-4b_2\cos\theta)$. Since $|b_1|>4b_2$, $h>0$ when $\theta\neq 0$.  Hence, $h(\theta)>0$ for $0<\theta\le \pi/2$.

\subsection{Analysis of the FODE}

\begin{lmm}
Suppose $b=\{b_n\}$ is a sequence satisfying $\sum_{n=0}^{\infty}b_n=0$
and $b_0>0$. Suppose further that $b_n=\frac{1}{\Gamma(-\gamma)}n^{-1-\gamma}(1+O(\frac{1}{n}))$, where $0<\gamma<1$. Let $F(z)$ be its generating function $F(z)=\sum_{n=0}^{\infty}b_n z^n$. Then, 
\begin{enumerate}
\item $$
F(z)-(1-z)^{\gamma}=(1-z)G(z)
$$
where $G(z)$ is bounded in the unit disk.
\item Let $b^{(-1)}$ be the convolution inverse of $b$. Its generatating function $F(z)^{-1}\sim (1-z)^{-\gamma}$ and thus $b^{(-1)}_n=[z^n]F(z)^{-1}\sim \frac{1}{\Gamma(\gamma)}n^{\gamma-1}$ 
\end{enumerate}
\end{lmm}
\begin{proof}
Let $H(z)=F(z)-(1-z)^{\gamma}$. Then, $H(1)=0$. Consider the Taylor series of $H$: $$
H(z)=\sum_{n=0}^{\infty}d_n z^n.
$$
By the asymptotic behavior of $b$ and the asymptotic behavior of the coefficients of $(1-z)^{\gamma}$, we find
$$d_n=O(\frac{1}{n^{2+\gamma}}).$$ 

Now consider $$
\frac{H(z)}{1-z}=\frac{H(z)-H(1)}{1-z}=\sum \frac{d_n(z^n-1)}{1-z}.
$$
Then, for each term, 
$|d_nz^n-d_n|(1-z)^{-1}\le Cnd_n$ where $C$ is independent of $n$ and $z\in D(0,1)$. Since $\sum n|d_n|$ converges. Hence, $G(z)=H(z)/(1-z)$ is bounded. The first claim is proved.


Using the first claim, we find that $$
F(z)^{-1}=(1-z)^{-\gamma}\frac{1}{1+(1-z)^{1-\gamma}G(z)}
$$
This implies that as $z\sim 1$, $F(z)^{-1}\sim (1-z)^{-\gamma}$.

By the lemma in the book of `analytical combinartorics', we have 
$b^{(-1)}_n=[z^n]F(z)^{-1}\sim \frac{1}{\Gamma(\gamma)}n^{\gamma-1}$.
\end{proof}
\begin{rmk}
Actually, we should have $G(1)=0$ also. Further, it should be the case that 
$$
F(z)-(1-z)^{\gamma}=(1-z)^{1+\gamma-\epsilon}G(z)
$$
and $G$ is still bounded, but we don't need these.
\end{rmk}

The following discrete Gronwall inequality is important:
\begin{lmm}
discrete gronwall
\end{lmm}

Now, we prove an important lemma
\begin{lmm}
Suppose $b=\{b_n\}$ is a sequence satisfying the properties above. Let $E=\{E^n\}$ be a nonnegative sequence. Let $b^{(-1)}$ be the convolution inverse of $b$, which has generating function $1/F(z)$. Let $k=T/N$ where $N$ is a big integer.

If one of the following two conditions is satisfied:

 (i). $\exists \gamma_n\ge 0,\beta\ge 0$, $\gamma_n\le Ck^{-\gamma}(1+n)^{-\gamma}$
 $$
k^{-\gamma}|(b*E)_{n+1}|\le \gamma_n+\beta E^n,\forall 0\le n\le N.
$$

(ii). $b^{(-1)}_n\ge 0$ and $\exists \gamma_n\ge0, \beta\ge0$, $\gamma_n\le Ck^{-\gamma}(1+n)^{-\gamma}$
$$
k^{-\gamma}(b*E)_{n+1}\le \gamma_n+\beta E^n,\forall 0\le n\le N.
$$
Then, $$
E^n\le C(T)E_{\gamma}(C_1(nk)^{\gamma}),\ \forall 0\le n\le N.
$$
\end{lmm}
\begin{proof}
Let $H_n=k^{-\gamma}(b*E)_{n}$ and consider the sequence $H=\{H_n\}$. Then clearly, we have $E^n=k^{\gamma}(b^{(-1)}*H)_n$ for all $n\ge 0$.

By the asymptotic behavior of $b^{(-1)}_n$, there exists $C$ independent of $n$ such that $|b^{(-1)}_n|\le C \frac{1}{\Gamma(\gamma)}(n+1)^{\gamma-1}$ for all $n\ge 0$.

Now, consider only $0\le n\le N$:

If the first case happens, we have 
\begin{gather*}
E^n\le k^{\gamma}\sum_{i=0}^n |b_{n-j}^{(-1)}| |H_{j}|
\le |b_n^{(-1)}|b_0E_0+Ck^{\gamma}\sum_{j=1}^{n} \frac{1}{\Gamma(\gamma)}(n-j+1)^{\gamma-1}
(\gamma_{j-1}+\beta E^{j-1})\\
\le C+k^{\gamma}\frac{C_1}{\Gamma(\gamma)}\sum_{j=0}^{n-1}(n-j)^{\gamma-1}E^j,\ \forall 0\le n\le N
\end{gather*}
Here, we have used the fact $k^{\gamma}\sum (n-j)^{\gamma-1}k^{-\gamma}(1+j)^{-\gamma}\sim const$

if the second case happens, then, we have 
\begin{gather*}
E^n\le k^{\gamma}\sum_{j=0}^n b_{n-j}^{(-1)} H_{j}
\le b_n^{(-1)}b_0E_0+Ck^{\gamma}\sum_{j=1}^{n} \frac{1}{\Gamma(\gamma)}(n-j+1)^{\gamma-1}
(\gamma_{j-1}+\beta E^{j-1})\\
\le C+\frac{C_1}{\Gamma(\gamma)}\sum_{j=0}^{n-1}(n-j)^{\gamma-1}E^j,\ \forall 0\le n\le N
\end{gather*}
The discrete Gronwall inequality guarantees that $E^n$ has the bound as indicated.
\end{proof}

We now prove theorem...
\begin{proof}

Now considering left hand side. The issue appears since $b_2>0$ if $\gamma>\gamma_0$.

The key point is to write 
\begin{gather*}
b_0u^{n+1}+b_1u^n+b_2u^{n-1}
=-\frac{b_1}{2}(\frac{3}{2}u^{n+1}-2u^n+\frac{1}{2}u^{n-1})
+(b_0+\frac{3}{4}b_1)u^{n+1}+(b_2+\frac{b_1}{4})u^{n-1}
\end{gather*}
Define the new sequence $c_0^{n+1}=b_0+\frac{3}{4}b_1>0$, $c_1^{n+1}=0$,
$c_2^{n+1}=b_2+\frac{1}{4}b_2<0$, $c_m^{n+1}=b_m<0, m\ge 3$. $\sum_m c_m^{n+1}=0$ still holds. We also use $c$ to mean the limiting sequence for $c^{n+1}$. By the technique used in...
\begin{gather*}
\langle u^{n+1}, \sum_m c_m^{n+1}u^{n+1-m}\rangle\ge 
\frac{1}{2}\sum_m c_m^{n+1}\|u^{n+1-m}\|^2.
\end{gather*}
Further, 
\begin{multline*}
\langle u^{n+1}, \frac{3}{2}u^{n+1}-2u^n+\frac{1}{2}u^{n-1}\rangle
\ge \frac{1}{4}(\|u^{n+1}\|_2^2-\|u^n\|_2^2+\|2u^{n+1}-u^n\|_2^2-\|2u^n-u^{n-1}\|_2^2)
\end{multline*}
Since $\langle u^{n+1}, (\mathcal{D}_k u)^{n+1}\rangle\le \lambda(\|u^n\|^2+\|u^{n+1}\|^2)$ and $-b_1>0$, we have
\begin{multline*}
-\frac{b_1}{8}k^{-\gamma}(\|u^{n+1}\|_2^2-\|u^n\|_2^2+\|2u^{n+1}-u^n\|_2^2-\|2u^n-u^{n-1}\|_2^2)\\
+\frac{1}{2}k^{-\gamma}\left(c_0\|u^{n+1}\|^2
+c_2\|u^{n-1}\|^2
+\sum_{m=3}^{n+1}c_m\|u^{n+1-m}\|^2\right)\\
\le k^{-\gamma}(b_{n}-b^{n+1}_n)\|u^1\|^2
+k^{-\gamma}(b_{n+1}-b_{n+1}^{n+1})\|u^0\|^2+\lambda(\|u^n\|^2+\|u^{n-1}\|^2)
\end{multline*}

Here, it is important to notice that 
\begin{gather*}
b_n^{n+1}-b_n=\frac{1}{2}n^{1-\gamma}+\frac{1}{2}(n+1)^{1-\gamma}
+\frac{1}{2-\gamma}n^{2-\gamma}
-\frac{1}{2-\gamma}(n+1)^{2-\gamma}=O(n^{-1-\gamma})\\
b_{n+1}^{n+1}-b_{n+1}=O(n^{-\gamma})
\end{gather*}
Set $F_n=\frac{1}{n+1}\sum_{m=0}^n\|u^m\|_2^2$.
Taking the summation to $n+1$ and diving by $n+2$, we have
\begin{multline*}
-\frac{b_1}{8(n+2)}k^{-\gamma}((\|u^{n+1}\|_2^2-\|u^0\|_2^2+\|2\rho^{n+1}-\rho^n\|_2^2-\|2\rho^1-\rho^{0}\|_2^2)\\
+\frac{1}{2}k^{-\gamma}\left(c_0F_{n+1}
+c_2\frac{n}{n+2}F_{n-1}
+\sum_{m=3}^{n+1}c_m\frac{n+2-m}{n+2}F_{n+1-m}\right)\\
\le k^{-\gamma}\frac{1}{n+2}\sum_{m=0}^{n+1}((b_{m}-b^{m+1}_m))\|\rho_1\|^2
+k^{-\gamma}\frac{1}{n+2}\sum_{m=0}^{m+1}(b_{m+1}-b_{m+1}^{m+1})\|\rho_0\|^2
+2\lambda F_n
\end{multline*}
Direct estimation tells us that $\frac{1}{n+1}\sum_{n=0}^{n+1}((b_{n}-b^{n+1}_n))\|\rho_1\|^2=O(n^{-\gamma})$ and $\frac{1}{n+1}\sum_{n=0}^{n+1}(b_{n+1}-b_{n+1}^{n+1})=O(n^{-\gamma})$.
Noting the signs of the coefficients, we have
\begin{multline*}
\frac{1}{2}k^{-\gamma}\left(c_0F_{n+1}+c_2F_{n-1}
+\sum_{m=3}^{n+1}c_mF_{n+1-m}\right)\le \\
-\frac{|b_1|}{8|n+2|k^{\gamma}}\|u^{n+1}\|^2+
C_2k^{-\gamma}(n+1)^{-\gamma}(\|u^0\|_2^2+\|u^1\|_2^2)+2\lambda F_n.
\end{multline*}
Using Lemma ..., we can show that $F$ is bounded since $c=(b_0+\frac{3}{4}b_1,0, b_2+\frac{b_1}{4}, b_3,\ldots)$ satisfies the conditions in the lemma.
\end{proof}
The condition on $A_k$ says that $A_k$ should have a kind of negativity. For example, if $A_k(u)=-\delta u^{n+1}+B(u^{n-1}, u^n)$ where $\delta>0$ and $|B(u^{n-1}, u^n)|\le C(\|u^{n-1}\|+\|u^n\|)$, then the condition is satisfied. 

\begin{cor}
The implicit second order scheme for $D_c^{\gamma}u=\lambda u, \lambda\le 0$ is $l^2$-stable.
\end{cor}
This agrees with the $A$-stability analysis.


Using the tools above, we can actually show some results about the first order scheme:

We denote the first order scheme as $$
(\mathscr{D}_ku)^{n+1}=k^{-\gamma}(a^{n+1}*u)_{n+1}.
$$

As we have seen, the difference scheme is $(b^{n+1}*u)_{n+1}$ where $b^{n+1}$ is a sequence with length $n+1$. 

\begin{thm}
If $E^n$ is a nonnegative sequence satisfying $$
(\mathscr{D}_kE)^{n+1}\le \lambda E^n
$$
then $$
E^n\le E^0 C(T)E_{\gamma}(C_1(nk)^{\gamma}), \forall 0\le n\le N.
$$
\end{thm}

\begin{proof}
Let $a=(a_0, a_1,\ldots)$ be the limit sequence.

As we know $a_0>0$ and $a_m<0$ for all $m\ge 1$. Then, we find that all components of $a^{(-1)}$ are positive.

By what has been computed, $a^{n+1}$ is exactly the same as the subsequence of $a$ for the first $n$ components. $$
k^{-\gamma}(a^{n+1}*E)_{n+1}=k^{-\gamma}(a*E)_{n+1}+k^{-\gamma}(a_{n+1}^{n+1}-a_{n+1})E^0\le CE^{n},\ \forall 0\le n\le N.
$$

Note that $a_{n+1}^{n+1}-a_{n+1}=\sum_{m=n+2}^{\infty}a_m=O(n^{-\gamma})$.

Applying the variant version of the lemma above shows that the first order KS scheme is stable.
\end{proof}

\begin{cor}
Consider the FODE $D_c^{\gamma}u=A(u)$. Suppose $\mathscr{A}_ku$ is a first order approximation for $A(u)$ such that $\exists\lambda\ge 0$, $$
\langle u^{n+1}, (\mathscr{A}_ku)^{n+1}\rangle\le \lambda(u^{n})^2.
$$
The scheme given by
$$
(\mathcal{D}_ku)^{n+1}=(\mathscr{A}_ku)^{n+1}, \ n\ge 1
$$
is stable.
\end{cor}

\begin{cor}
Consider the FODE $$
D_c^{\gamma}u=\lambda u.
$$
The first order explicit scheme given by $$
(\mathscr{D}_ku)^{n+1}=\lambda u^n,
$$
is stable for $\lambda\ge 0$.

The first order implicit scheme 
$$
(\mathscr{D}_ku)^{n+1}=\lambda u^{n+1},
$$
is stable for $\lambda\le 0$.
\end{cor}
\begin{proof}
Consider the explicit scheme and $\lambda\ge 0$. Since both the FODE and the scheme are linear. We can consider $u_0\ge 0$.
By the sign of the coefficients, we find that $u^n$ are all nonnegative.
The theorem then implies ....

For the implicit scheme, since $\langle u^{n+1}, \lambda u^{n+1}\rangle\le 0$. Using Corollary..., we find that ...
\end{proof}

\subsection{Analysis of fractional KS}

For the FKS, the continuous model,

Integrating by parts, $$
\frac{u(t)-u(0)}{t^{\gamma}}+\gamma\int_0^t\frac{u(t)-u(s)}{(t-s)^{1+\gamma}}ds
=\ldots
$$
It is clear that the difference scheme is a kind of approximation for this.
$b_0\approx \frac{1}{t^{\gamma}}+\gamma\int_0^{t-k}\frac{1}{(t-s)^{1+\gamma}}ds$, etc..

Since $$
\langle u, \Delta u-\nabla\cdot(\rho\nabla c)\rangle
=-\|\nabla u\|_2^2+\frac{1}{2}\|\rho^3\|_1,
$$
and $\|\rho^3\|_1\le C\|\rho\|_1\|\nabla\rho\|_2^2$ by Gargolidardo-Nirenberg inequality for dimension $2$, we see that if the initial mass is small, then, 
 $$
\langle u(t), \frac{u(t)-u(0)}{t^{\gamma}}+\gamma\int_0^t\frac{u(t)-u(s)}{(t-s)^{1+\gamma}}ds\rangle\le 0. 
$$
Since $a(a-b)\ge \frac{1}{2}(a^2-b^2)$, then $\|u(t)\|_2^2$ decays.


For the semi-discrete scheme, we just verify the conditions....



We just verify the conditions in....

\section{Numerical results}
\subsection{FODE}
stability region plot with variant $n$. \\

convergence order test for $D^\gamma_c u = \lambda u$ with BDF and variant BDF methods, where exact solution is available. \\


convergence order test for $D^\gamma_c u = f( u)$, where reference solution is computed with fine mesh.

\subsection{FKS}



\appendix

\section{The negativity of $P_m$}
\begin{lmm}
$P_m(\tau)\le 0$ for all $m\ge 1$ and $\tau\in I_m$.
\end{lmm}
\begin{proof}
\begin{multline*}
P_m(\tau)=(t^{m-1}-\tau)^2k^{2-\gamma}[\frac{1}{2}(n-m)^{1-\gamma}+\frac{1}{2-\gamma}(n-m)^{2-\gamma}]+\\
(-2)(t^m-\tau)^2k^{2-\gamma}[(n-m)^{1-\gamma}+\frac{1}{2-\gamma}(n-m)^{2-\gamma}]\\
+(t^{m+1}-\tau)^2k^{2-\gamma}[\frac{1}{2}(n-m+1)^{1-\gamma}+\frac{1}{2-\gamma}(n-m+1)^{2-\gamma}]\\
-\frac{2k^2}{2-\gamma}(t^{n+1}-\tau)^{2-\gamma}
\end{multline*}

$P_m(\tau=t^{m+1})=0$ and
$P_m(\tau=t^m)=k^2\int_{I_m}\frac{t^{m+1/2}-s}{(t^{n+1}-s)^{\gamma}}ds<0$.

We find that 
\begin{multline*}
(1-\gamma)P'=2(\tau-t^{m-1})k^{2-\gamma}[\frac{1}{2}(n-m)^{1-\gamma}+\frac{1}{2-\gamma}(n-m)^{2-\gamma}]+\\
(-2)2(\tau-t^m)k^{2-\gamma}[(n-m)^{1-\gamma}+\frac{1}{2-\gamma}(n-m)^{2-\gamma}]\\
+2(\tau-t^{m+1})k^{2-\gamma}[\frac{1}{2}(n-m+1)^{1-\gamma}+\frac{1}{2-\gamma}(n-m+1)^{2-\gamma}]\\
+2k^2(t^{n+1}-\tau)^{1-\gamma}
\end{multline*}
We can show that $P'(t^m)<0$. $P'(t^{m+1})=0$. 

Further, 
\begin{multline*}
(1-\gamma)P''(t^{m+1})/k^{2-\gamma}=
-3(n-m)^{1-\gamma}-\frac{2}{2-\gamma}(n-m)^{2-\gamma}+\\
+[(n-m+1)^{1-\gamma}+\frac{2}{2-\gamma}(n-m+1)^{2-\gamma}]
-2(1-\gamma)(n-m)^{-\gamma}\\
=(n-m+1)^{1-\gamma}-3(n-m)^{1-\gamma}
+2\int_{n-m}^{n-m+1}x^{1-\gamma}dx-2(1-\gamma)(n-m)^{-\gamma}
\end{multline*}
Using the inequality $x^{1-\gamma}\le (n-m)^{1-\gamma}+(1-\gamma)(n-m)^{-\gamma}(x-(n-m))$ since the function is concave, we find $$
(1-\gamma)P''(t^{m+1})/k^{2-\gamma}\le (n-m+1)^{1-\gamma}-(n-m)^{1-\gamma}
-(1-\gamma)(n-m)^{-\gamma}\le 0.
$$
Hence, $P''$ could be positive on $\tau\in (t^m, t_0)$ and negative on $(t_0, t^{m+1})$ or all negative on $(t^m, t^{m+1})$. Together with the fact $P'(t^{m+1})=0$, we know either $P$ either first decreases and then increases or increases for all time. Since $P(t^{m+1})=0$ and $P(t^m)<0$, we find that $P\le 0,\tau\in (t^m, t^{m+1})$.
\end{proof}
\end{document}
